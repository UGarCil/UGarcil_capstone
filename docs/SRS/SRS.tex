% THIS DOCUMENT IS TAILORED TO REQUIREMENTS FOR SCIENTIFIC COMPUTING.  IT SHOULDN'T
% BE USED FOR NON-SCIENTIFIC COMPUTING PROJECTS
\documentclass[12pt]{article}

\usepackage{amsmath, mathtools}
\usepackage{amsfonts}
\usepackage{amssymb}
\usepackage{graphicx}
\usepackage{colortbl}
\usepackage{xr}
\usepackage{hyperref}
\usepackage{longtable}
\usepackage{xfrac}
\usepackage{tabularx}
\usepackage{float}
\usepackage{siunitx}
\usepackage{booktabs}
\usepackage{caption}
\usepackage{pdflscape}
\usepackage{afterpage}

\usepackage[round]{natbib}

\usepackage{hyperref}
%\usepackage{refcheck}

\hypersetup{
    bookmarks=true,         % show bookmarks bar?
      colorlinks=true,       % false: boxed links; true: colored links
    linkcolor=red,          % color of internal links (change box color with linkbordercolor)
    citecolor=green,        % color of links to bibliography
    filecolor=magenta,      % color of file links
    urlcolor=cyan           % color of external links
}

\input{../Comments}
%% Common Parts

\newcommand{\progname}{SubLiMat} % PUT YOUR PROGRAM NAME HERE
\newcommand{\authname}{Uriel Garcilazo Cruz} % AUTHOR NAMES                  

\usepackage{hyperref}
    \hypersetup{colorlinks=true, linkcolor=blue, citecolor=blue, filecolor=blue,
                urlcolor=blue, unicode=false}
    \urlstyle{same}
                                


% For easy change of table widths
\newcommand{\colZwidth}{1.0\textwidth}
\newcommand{\colAwidth}{0.13\textwidth}
\newcommand{\colBwidth}{0.82\textwidth}
\newcommand{\colCwidth}{0.1\textwidth}
\newcommand{\colDwidth}{0.05\textwidth}
\newcommand{\colEwidth}{0.8\textwidth}
\newcommand{\colFwidth}{0.17\textwidth}
\newcommand{\colGwidth}{0.5\textwidth}
\newcommand{\colHwidth}{0.28\textwidth}

% Used so that cross-references have a meaningful prefix
\newcounter{defnum} %Definition Number
\newcommand{\dthedefnum}{GD\thedefnum}
\newcommand{\dref}[1]{GD\ref{#1}}
\newcounter{datadefnum} %Datadefinition Number
\newcommand{\ddthedatadefnum}{DD\thedatadefnum}
\newcommand{\ddref}[1]{DD\ref{#1}}
\newcounter{theorynum} %Theory Number
\newcommand{\tthetheorynum}{TM\thetheorynum}
\newcommand{\tref}[1]{TM\ref{#1}}
\newcounter{tablenum} %Table Number
\newcommand{\tbthetablenum}{TB\thetablenum}
\newcommand{\tbref}[1]{TB\ref{#1}}
\newcounter{assumpnum} %Assumption Number
\newcommand{\atheassumpnum}{A\theassumpnum}
\newcommand{\aref}[1]{A\ref{#1}}
\newcounter{goalnum} %Goal Number
\newcommand{\gthegoalnum}{GS\thegoalnum}
\newcommand{\gsref}[1]{GS\ref{#1}}
\newcounter{instnum} %Instance Number
\newcommand{\itheinstnum}{IM\theinstnum}
\newcommand{\iref}[1]{IM\ref{#1}}
\newcounter{reqnum} %Requirement Number
\newcommand{\rthereqnum}{R\thereqnum}
\newcommand{\rref}[1]{R\ref{#1}}
\newcounter{nfrnum} %NFR Number
\newcommand{\rthenfrnum}{NFR\thenfrnum}
\newcommand{\nfrref}[1]{NFR\ref{#1}}
\newcounter{lcnum} %Likely change number
\newcommand{\lthelcnum}{LC\thelcnum}
\newcommand{\lcref}[1]{LC\ref{#1}}

\usepackage{fullpage}

\newcommand{\deftheory}[9][Not Applicable]
{
\newpage
\noindent \rule{\textwidth}{0.5mm}

\paragraph{RefName: } \textbf{#2} \phantomsection 
\label{#2}

\paragraph{Label:} #3

\noindent \rule{\textwidth}{0.5mm}

\paragraph{Equation:}

#4

\paragraph{Description:}

#5

\paragraph{Notes:}

#6

\paragraph{Source:}

#7

\paragraph{Ref.\ By:}

#8

\paragraph{Preconditions for \hyperref[#2]{#2}:}
\label{#2_precond}

#9

\paragraph{Derivation for \hyperref[#2]{#2}:}
\label{#2_deriv}

#1

\noindent \rule{\textwidth}{0.5mm}

}

\DeclareSIUnit{\cellunit}{cu}  % 'cu' is the symbol that will appear



\begin{document}

\title{Software Requirements Specification for \progname: SubLiMat} 
\author{Uriel Garcilazo Cruz}
\date{\today}
	
\maketitle

~\newpage

\pagenumbering{roman}

\tableofcontents

~\newpage

\section*{Revision History}

\begin{tabularx}{\textwidth}{p{3cm}p{2cm}X}
\toprule {\bf Date} & {\bf Version} & {\bf Notes}\\
\midrule
February 1 2025 & 1.0 & Document Creation\\
% Date 2 & 1.1 & Notes\\
\bottomrule
\end{tabularx}

% ~\\
% \plt{This template is intended for use by CAS 741.  For CAS 741 the template
% should be used exactly as given, except the Reflection Appendix can be deleted.
% For the capstone course it is a source of ideas, but shouldn't be followed
% exactly.  The exception is the reflection appendix.  All capstone SRS documents
% should have a reflection appendix.}

~\newpage

\section{Reference Material}

This section records information for easy reference.

\subsection{Table of Units}

Throughout this document SI (Syst\`{e}me International d'Unit\'{e}s) is NOT employed
as the unit system.  In addition to the basic units from SI, several units
are described under the symbols section below.

~\newline

\renewcommand{\arraystretch}{1.2}
\begin{center}
  \begin{tabular}{l l l} 
    \toprule		
    \textbf{symbol} & \textbf{unit} & \textbf{SI}\\
    \midrule 
    
    \bottomrule
  \end{tabular}
\end{center}

  %	\caption{Provide a caption}
%\end{table}

% \plt{Only include the units that your SRS actually uses.}

% \plt{Derived units, like newtons, pascal, etc, should show their derivation
%     (the units they are derived from) if their constituent units are in the
%     table of units (that is, if the units they are derived from are used in the
%     document).  For instance, the derivation of pascals as
%     $\si{\pascal}=\si{\newton\per\square\meter}$ is shown if newtons and m are
%     both in the table.  The derivations of newtons would not be shown if kg and
%     s are not both in the table.}

% \plt{The symbol for units named after people use capital letters, but the name
%   of the unit itself uses lower case.  For instance, pascals use the symbol Pa,
%   watts use the symbol W, teslas use the symbol T, newtons use the symbol N,
%   etc.  The one exception to this is degree Celsius.  Details on writing metric
%   units can be found on the 
%   \href{https://www.nist.gov/pml/weights-and-measures/writing-metric-units}
%   {NIST} web-page.}

\subsection{Table of Symbols}

Throughout this document the standard HGSV nomenclature is employed as the unit system.
Additional units that are unique tu this document are prefixed with *.

\begin{center}
  \begin{tabular}{l l l} 
    \toprule		
    \textbf{symbol} & \textbf{unit} & \textbf{HGSV}\\
    \midrule 
    *qa & alignment quality & fundamental unit of alignment quality
    \\
    bp & base pair & fundamental unit of genetic sequence length
    \\
    Kb & kilobase unit & one thousand base pairs
    \\
    \bottomrule
  \end{tabular}
  \end{center}

\vspace{1em}

The table that follows summarizes the symbols used in this document along with
their units.  The choice of symbols was made to be consistent with the bioinformatics
literature and with the existing internationally recognized standard for the description
 of DNA, RNA and protein reading frames. The symbols are listed in alphabetical order.

\renewcommand{\arraystretch}{1.2}
%\noindent \begin{tabularx}{1.0\textwidth}{l l X}
\noindent \begin{longtable*}{l l p{12cm}} \toprule
\textbf{symbol} & \textbf{unit} & \textbf{description}\\
\midrule 
  F & -- & Comparative alignment between two sequences
  \\
  g & qa & penalty associated with a gap in the alignment of two sequences
  \\
  $N_A$ & bp & Adenine nitrogenous base, element of the purine family of nucleotides with an amine group in Carbon 6 of its pyrimidine ring
  \\
  $N_C$ & bp & Cytosine nitrogenous base, element of the pyrimidine family of nucleotides with a no methylated carbons making part of its pyrimidine ring
  \\
  $N_G$ & bp & Guanine nitrogenous base, element of the purine family of nucleotides with an amine group on Carbon 2 and a carbonyl group on Carbon 6 of its pyrimidine ring
  \\
  $N_T$ & bp & Tymine nitrogenous base, element of the pyrimidine family of nucleotides with a methyl group in Carbon 5 of its pyrimidine ring
  \\
  $Q_{AB}$ & qa & A collection of base pairs representing a genetic sequence to be compared with another sequence $SEQ_A$
  \\
  S & qa & Substitution matrix used to score the alignment of two sequences
  \\
  $SEQ_A$ & Kb & A collection of base pairs representing a genetic sequence to be compared with another sequence $SEQ_B$
  \\
  $SEQ_B$ & Kb & A collection of base pairs representing a genetic sequence to be compared with another sequence $SEQ_A$
  \\
  SNP & bp & single nucleotide polymorphism; variation in a single base pair in DNA sequence
  \\
  $T_I$ & qa & A transition occurrying between nucleotides of the same nitrogenous base families
  \\
  $T_V$ & qa & A transition occurrying between nucleotides of different nitrogenous base families
  \\
\bottomrule
\end{longtable*}
% \plt{Use your problems actual symbols.  The si package is a good idea to use for
  % units.}

\subsection{Abbreviations and Acronyms}

\renewcommand{\arraystretch}{1.2}
\begin{tabular}{l l} 
  \toprule		
  \textbf{symbol} & \textbf{description}\\
  \midrule 
  A & Assumption\\
  DD & Data Definition\\
  GD & General Definition\\
  GS & Goal Statement\\
  IM & Instance Model\\
  LC & Likely Change\\
  PS & Physical System Description\\
  R & Requirement\\
  SRS & Software Requirements Specification\\
  \progname{} & Substitution Matrix benchmarking with pairwise alignment\\
  TM & Theoretical Model\\
  \bottomrule
\end{tabular}\\

% \plt{Add any other abbreviations or acronyms that you add}

% \subsection{Mathematical Notation}

% \plt{This section is optional, but should be included for projects that make use
%   of notation to convey mathematical information.  For instance, if typographic
%   conventions (like bold face font) are used to distinguish matrices, this
%   should be stated here.  If symbols are used to show mathematical operations,
%   these should be summarized here.  In some cases the easiest way to summarize
%   the notation is to point to a text or other source that explains the
%   notation.}

% \plt{This section was added to the template because some students use very
%   domain specific notation.  This notation will not be readily understandable to
%   people outside of your domain.  It should be explained.}

% \newpage

% \pagenumbering{arabic}

% \plt{This SRS template is based on \citet{SmithAndLai2005, SmithEtAl2007,
%   SmithAndKoothoor2016}.  It will get you started.  You should not modify the
%   section headings, without first discussing the change with the course
%   instructor.  Modification means you are not following the template, which
%   loses some of the advantage of a template, especially standardization.
%   Although the bits shown below do not include type information, you may need to
%   add this information for your problem.  If you are unsure, please can ask the
%   instructor.}

% \plt{Feel free to change the appearance of the report by modifying the LaTeX
%   commands.}

% \plt{This template document assumes that a single program is being documented.
%   If you are documenting a family of models, you should start with a commonality
%   analysis.  A separate template is provided for this.  For program
%   families you should look at \cite{Smith2006, SmithMcCutchanAndCarette2017}.
%   Single family member programs are often programs based on a single physical
%   model.  General purpose tools are usually documented as a family.  Families of
%   physical models also come up.}

% \plt{The SRS is not generally written, or read, sequentially.  The SRS is a
%   reference document.  It is generally read in an ad hoc order, as the need
%   arises.  For writing an SRS, and for reading one for the first time, the
%   suggested order of sections is:
% \begin{itemize}
% \item Goal Statement
% \item Instance Models
% \item Requirements
% \item Introduction
% \item Specific System Description
% \end{itemize}
% }

% \plt{Guiding principles for the SRS document:
% \begin{itemize}
% \item Do not repeat the same information at the same abstraction level.  If
%   information is repeated, the repetition should be at a different abstraction
%   level.  For instance, there will be overlap between the scope section and the
%   assumptions, but the scope section will not go into as much detail as the
%   assumptions section.
% \end{itemize}
% }

% \plt{The template description comments should be disabled before submitting this
%   document for grading.}

% \plt{You can borrow any wording from the text given in the template.  It is part
%   of the template, and not considered an instance of academic integrity.  Of
%   course, you need to cite the source of the template.}

% \plt{When the documentation is done, it should be possible to trace back to the
%   source of every piece of information.  Some information will come from
%   external sources, like terminology.  Other information will be derived, like
%   General Definitions.}

% \plt{An SRS document should have the following qualities: unambiguous,
%   consistent, complete, validatable, abstract and traceable.}

% \plt{The overall goal of the SRS is that someone that meets the Characteristics
%   of the Intended Reader (Section~\ref{sec_IntendedReader}) can learn,
%   understand and verify the captured domain knowledge.  They should not have to
%   trust the authors of the SRS on any statements.  They should be able to
%   independently verify/derive every statement made.}

\section{Introduction}

Substitution matrices are critical assumptions that greatly impact studies
in the area of comparative biology, yet, benchmarking these matrices is a 
laborious task.

The following section contains an overview of the Software Requirements Specification (SRS)
for a substitution matrix benchmark tool via pairwise alignment. The program is referred to 
as SubLiMat. The purpose of this section is to characterize the purpose, scope of Requirements, 
characteristics of Intended Reader, and Organization of the SRS document.

% \plt{The introduction section is written to introduce the problem.  It starts
%   general and focuses on the problem domain. The general advice is to start with
% a paragraph or two that describes the problem, followed by a ``roadmap''
% paragraph.  A roadmap orients the reader by telling them what sub-sections to
% expect in the Introduction section.}

\subsection{Purpose of Document}

The purpose of this document is to provide a detailed and standardized characterization of 
the elements, theoretical and operational, that surround the SubLiMat software. Such 
elements include goals, assumptions, and theoretical and instanced models that describe 
the scientific basis of the software. Moreover, the document is intended to be used as a guide to detail 
the unique characteristics of the software to improve on its verifiability and correctness. 

% \plt{This section summarizes the purpose of the SRS document.  It does not focus
%   on the problem itself.  The problem is described in the ``Problem
%   Description'' section (Section~\ref{Sec_pd}).  The purpose is for the document
%   in the context of the project itself, not in the context of this course.
%   Although the ``purpose'' of the document is to get a grade, you should not
%   mention this.  Instead, ``fake it'' as if this is a real project.  The purpose
%   section will be similar between projects.  The purpose of the document is the
%   purpose of the SRS, including communication, planning for the design stage,
%   etc.}

\subsection{Scope of Requirements} 

The scope of the requirements for the SubLiMat software includes the evaluation of 
moderate-sized genetic sequences with similar dimensions. 

% \plt{Modelling the real world requires simplification.  The full complexity of
%   the actual physics, chemistry, biology is too much for existing models, and
%   for existing computational solution techniques.  Rather than say what is in
%   the scope, it is usually easier to say what is not.  You can think of it as
%   the scope is initially everything, and then it is constrained to create the
%   actual scope.  For instance, the problem can be restricted to 2 dimensions, or
%   it can ignore the effect of temperature (or pressure) on the material
%   properties, etc.}  

% \plt{The scope section is related to the assumptions section
%   (Section~\ref{sec_assumpt}).  However, the scope and the assumptions are not
%   at the same level of abstraction.  The scope is at a high level.  The focus is
%   on the ``big picture'' assumptions.  The assumptions section lists, and
%   describes, all of the assumptions.}

% \plt{The scope section is relevant for later determining typical values of inputs. The scope should make it clear what inputs are reasonable to expect. This is a distinction between scope and context (context is a later section).  Scope affects the inputs while context affects how the software will be used.}

\subsection{Characteristics of Intended Reader} \label{sec_IntendedReader}

The intended readers of this documentation should have a general understanding of genetics, 
equivalent or higher to a highschool level. Although not necessary, the document may benefit
from a reader who possesses a basic understanding of comparative biology equivalent to first 
year university level or higher.  

% \plt{This section summarizes the skills and knowledge of the readers of the
%   SRS.  It does NOT have the same purpose as the ``User Characteristics''
%   section (Section~\ref{SecUserCharacteristics}).  The intended readers are the
%   people that will read, review and maintain the SRS.  They are the people that
%   will conceivably design the software that is intended to meet the
%   requirements.  The user, on the other hand, is the person that uses the
%   software that is built.  They may never read this SRS document.  Of course,
%   the same person could be a ``user'' and an ``intended reader.''}

% \plt{The intended reader characteristics should be written as unambiguously and
%   as specifically as possible.  Rather than say, the user should have an
%   understanding of physics, say what kind of physics and at what level.  For
%   instance, is high school physics adequate, or should the reader have had a
%   graduate course on advanced quantum mechanics?}

\subsection{Organization of Document}

The structure of this document follows the standard template for an SRS document.
As presented by \href{https://jacquescarette.github.io/Drasil/examples/swhsnopcm/SRS/HTML/SWHSNoPCM_SRS.html}
{Jegatheesan \& Smith, 2019} in their SRS example for this section:

\begin{quote}
The organization of this document follows the template for an SRS for scientific 
computing software proposed by \cite{koothoor2013}, \cite{smithLai2005}, 
\cite{smithEtAl2007}, and \cite{smithKoothoor2016}. The presentation follows 
the standard pattern of presenting goals, theories, definitions, and assumptions.
\dots


The goal statements are refined to the theoretical models and the theoretical 
models to the instance models. The instance model to be solved is referred to 
as IM:eBalanceOnWtr. The instance model provides the Ordinary Differential 
Equation (ODE) that models the solar water heating system with no phase change 
material. SWHSNoPCM solves this ODE.
\end{quote}
% \plt{This section provides a roadmap of the SRS document.  It will help the
%   reader orient themselves.  It will provide direction that will help them
%   select which sections they want to read, and in what order.  This section will
%   be similar between project.}

\section{General System Description}

This section provides general information about the system.  It identifies the
interfaces between the system and its environment, describes the user
characteristics and lists the system constraints.
% \plt{This text can likely be
%   borrowed verbatim.}

% \plt{The purpose of this section is to provide general information about the
%   system so the specific requirements in the next section will be easier to
%   understand. The general system description section is designed to be
%   changeable independent of changes to the functional requirements documented in
%   the specific system description. The general system description provides a
%   context for a family of related models.  The general description can stay the
%   same, while specific details are changed between family members.}

\subsection{System Context}

% \plt{Your system context will include a figure that shows the abstract view of
%   the software.  Often in a scientific context, the program can be viewed
%   abstractly following the design pattern of Inputs $\rightarrow$ Calculations
%   $\rightarrow$ Outputs.  The system context will therefore often follow this
%   pattern.  The user provides inputs, the system does the calculations, and then
%   provides the outputs to the user.  The figure should not show all of the
%   inputs, just an abstract view of the main categories of inputs (like material
%   properties, geometry, etc.).  Likewise, the outputs should be presented from
%   an abstract point of view.  In some cases the diagram will show other external
%   entities, besides the user.  For instance, when the software product is a
%   library, the user will be another software program, not an actual end user.
%   If there are system constraints that the software must work with external
%   libraries, these libraries can also be shown on the System Context diagram.
%   They should only be named with a specific library name if this is required by
%   the system constraint.}

\begin{figure}[h!]
\begin{center}
 \includegraphics[width=0.8\textwidth]{sublimat_SystemContextFigure}
\caption{System Context}
\label{Fig_SystemContext} 
\end{center}
\end{figure}

% \plt{For each of the entities in the system context diagram its responsibilities
%   should be listed.  Whenever possible the system should check for data quality,
%   but for some cases the user will need to assume that responsibility.  The list
%   of responsibilities should be about the inputs and outputs only, and they
%   should be abstract.  Details should not be presented here.  However, the
%   information should not be so abstract as to just say ``inputs'' and
%   ``outputs''.  A summarizing phrase can be used to characterize the inputs.
%   For instance, saying ``material properties'' provides some information, but it
%   stays away from the detail of listing every required properties.}

\begin{itemize}
\item User Responsibilities:
\begin{itemize}
\item Provide genetic sequences of DNA
\item Provide meaningful genetic sequences presumed to share a common ancestor
\item Provide genetic sequences of similar dimensions
\end{itemize}
\item \progname{} Responsibilities:
\begin{itemize}
\item Detect data type mismatch, such as a string of characters instead of a
  floating point number
\item Determine if there exist any base pairs in the genetic sequences that are
  not part of the standard genetic code nomenclature for DNA
\item Calculate the alignment quality between two genetic sequences to produce outputs
\end{itemize}
\end{itemize}

% \plt{Identify in what context the software will typically be used.  Is it for
% exploration? education? engineering work? scientific work?. Identify whether it
% will be used for mission-critical or safety-critical applications.} \plt{This
% additional context information is needed to determine how much effort should be
% devoted to the rationale section.  If the application is safety-critical, the
% bar is higher.  This is currently less structured, but analogous to, the idea to
% the Automotive Safety Integrity Levels (ASILs) that McSCert uses in their
% automotive hazard analyses.}

\subsection{User Characteristics} \label{SecUserCharacteristics}

The end user of SubLiMat should have a general understanding of genetics and be familiar 
with the use of genetic sequences to make hypotheses of common ancestry.
% \plt{This section summarizes the knowledge/skills expected of the user.
%   Measuring usability, which is often a required non-function requirement,
%   requires knowledge of a typical user.  As mentioned above, the user is a
%   different role from the ``intended reader,'' as given in
%   Section~\ref{sec_IntendedReader}.  As in Section~\ref{sec_IntendedReader}, the
%   user characteristics should be specific an unambiguous.  For instance, ``The
%   end user of \progname{} should have an understanding of undergraduate Level 1
%   Calculus and Physics.''}

\subsection{System Constraints}

An alphabet of 4 letters, standardized to the genetic DNA nomenclature, should be used 
to represent the genetic sequences.

% \plt{System constraints differ from other type of requirements because they
%   limit the developers' options in the system design and they identify how the
%   eventual system must fit into the world. This is the only place in the SRS
%   where design decisions can be specified.  That is, the quality requirement for
%   abstraction is relaxed here.  However, system constraints should only be
%   included if they are truly required.}

\section{Specific System Description}


% This section first presents the problem description, which gives a high-level
% view of the problem to be solved.  This is followed by the solution characteristics
% specification, which presents the assumptions, theories, definitions and finally
% the instance models.  \plt{Add any project specific details that are relevant
%   for the section overview.}

\subsection{Problem Description} \label{Sec_pd}

\progname{} is intended to solve the uncertainties associated with the influence of 
substitution matrices on the alignment quality of genetic sequences.
% ... \plt{What problem does your program solve?
% The description here should be in the problem space, not the solution space.}

\subsubsection{Terminology and  Definitions}


% \plt{This section is expressed in words, not with equations.  It provide the
%   meaning of the different words and phrases used in the domain of the problem.
% The terminology is used to introduce concepts from the world outside of the
% mathematical model  The terminology provides a real world connection to give the
% mathematical model meaning.}

This subsection provides a list of terms that are used in the subsequent
sections and their meaning, with the purpose of reducing ambiguity and making it
easier to correctly understand the requirements:

\begin{itemize}

\item Substitution matrix: A square matrix that summarizes 
the rewards or penalties of moving from one base pair to another and 
expressed in units of alignment quality (qa)
\item Pairwise alignment: A pairwise alignment is the process of aligning two genetic 
sequences to identify regions of similarity 
\item Genetic sequence: A genetic sequence is a string of characters that represent 
the nucleotides of a DNA molecule and expressed in units of base pairs (bp)
\item Alignment quality: A measure of the quality of the alignment between two genetic 
sequences and express in units of alignment quality (qa)
\item gap: A gap is a space in the alignment of two genetic sequences that represents
a deletion or insertion of a base pair, and penalized with units of alignment quality (qa)

\end{itemize}

\subsubsection{Physical System Description} \label{sec_phySystDescrip}

% \plt{The purpose of this section is to clearly and unambiguously state the
%   physical system that is to be modelled. Effective problem solving requires a
%   logical and organized approach. The statements on the physical system to be
%   studied should cover enough information to solve the problem. The physical
%   description involves element identification, where elements are defined as
%   independent and separable items of the physical system. Some example elements
%   include acceleration due to gravity, the mass of an object, and the size and
%   shape of an object. Each element should be identified and labelled, with their
%   interesting properties specified clearly. The physical description can also
%   include interactions of the elements, such as the following: i) the
%   interactions between the elements and their physical environment; ii) the
%   interactions between elements; and, iii) the initial or boundary conditions.}

% \plt{The elements of the physical system do not have to correspond to an actual
% physical entity.  They can be conceptual.  This is particularly important when
% the documentation is for a numerical method. }

The physical system of \progname{}, as shown in Figure~\ref{fig:sequence_alignment},
includes the following elements:

\begin{itemize}

\item[PS1:] sequence $SEQ_A$ and sequence $SEQ_B$

\item[PS2:] Pairwise comparison matrix $F$ between the two sequences

\end{itemize}

% \plt{A figure here makes sense for most SRS documents}
\begin{figure}[h!]
  \begin{center}
  %\rotatebox{-90}
  {
   \includegraphics[width=0.5\textwidth]{physical_system_sublimat.png}
  }
  \caption{Pairwise comparison of genetic sequences}
  \label{fig:sequence_alignment}
  \end{center}
\end{figure}

\subsubsection{Goal Statements}

% \plt{The goal statements refine the ``Problem Description''
%   (Section~\ref{Sec_pd}).  A goal is a functional objective the system under
%   consideration should achieve. Goals provide criteria for sufficient
%   completeness of a requirements specification and for requirements
%   pertinence. Goals will be refined in Section “Instanced Models”
%   (Section~\ref{sec_instance}). Large and complex goals should be decomposed
%   into smaller sub-goals.  The goals are written abstractly, with a minimal
%   amount of technical language.  They should be understandable by non-domain
%   experts.}

\noindent Given two genetic sequences of size n, the goal statements are:

\begin{itemize}

\item[GS\refstepcounter{goalnum}\thegoalnum \label{generate_alignment}:] Generate 
scores ranking the quality of the alignment between two genetic sequences across
multiple substitution matrices
% \item[GS\refstepcounter{goalnum}\thegoalnum \label{generate_rank}:] Rank  
% multiple substitution matrices by the quality of the alignments they produce

\end{itemize}

\subsection{Solution Characteristics Specification}

% \plt{This section specifies the information in the solution domain of the system
%   to be developed. This section is intended to express what is required in
%   such a way that analysts and stakeholders get a clear picture, and the
%   latter will accept it. The purpose of this section is to reduce the problem
%   into one expressed in mathematical terms. Mathematical expertise is used to
%   extract the essentials from the underlying physical description of the
%   problem, and to collect and substantiate all physical data pertinent to the
%   problem.}

% \plt{This section presents the solution characteristics by successively refining
%   models.  It starts with the abstract/general Theoretical Models (TMs) and
%   refines them to the concrete/specific Instance Models (IMs).  If necessary
%   there are intermediate refinements to General Definitions (GDs).  All of these
%   refinements can potentially use Assumptions (A) and Data Definitions (DD).
%   TMs are refined to create new models, that are called GMs or IMs. DDs are not
%   refined; they are just used. GDs and IMs are derived, or refined, from other
%   models. DDs are not derived; they are just given. TMs are also just given, but
%   they are refined, not used.  If a potential DD includes a derivation, then
%   that means it is refining other models, which would make it a GD or an IM.}

% \plt{The above makes a distinction between ``refined'' and ``used.'' A model is
%   refined to another model if it is changed by the refinement. When we change a
%   general 3D equation to a 2D equation, we are making a refinement, by applying
%   the assumption that the third dimension does not matter. If we use a
%   definition, like the definition of density, we aren't refining, or changing
%   that definition, we are just using it.}

% \plt{The same information can be a TM in one problem and a DD in another.  It is
%   about how the information is used.  In one problem the definition of
%   acceleration can be a TM, in another it would be a DD.}

% \plt{There is repetition between the information given in the different chunks
%   (TM, GDs etc) with other information in the document.  For instance, the
%   meaning of the symbols, the units etc are repeated.  This is so that the
%   chunks can stand on their own when being read by a reviewer/user.  It also
%   facilitates reuse of the models in a different context.}

% \noindent \plt{The relationships between the parts of the document are show in
%   the following figure.  In this diagram ``may ref'' has the same role as
%   ``uses'' above.  The figure adds ``Likely Changes,'' which are able to
%   reference (use) Assumptions.}

\begin{figure}[H]
  \includegraphics[scale=0.9]{RelationsBetweenTM_GD_IM_DD_A.pdf}
\end{figure}

The instance models that govern \progname{} are presented in
Subsection~\ref{sec_instance}.  The information to understand the meaning of the
instance models and their derivation is also presented, so that the instance
models can be verified.

% \subsubsection{Types}

% \plt{This section is optional. Defining types can make the document easier to
% understand.}

% \subsubsection{Scope Decisions}

% \plt{This section is optional.}
% \subsubsection{Modelling Decisions}

% \plt{This section is optional.}

\subsubsection{Assumptions} \label{sec_assumpt}

% \plt{The assumptions are a refinement of the scope.  The scope is general, where
%   the assumptions are specific.  All assumptions should be listed, even those
%   that domain experts know so well that they are rarely (if ever) written down.}
% \plt{The document should not take for granted that the reader knows which
%   assumptions have been made. In the case of unusual assumptions, it is
%   recommended that the documentation either include, or point to, an explanation
%   and justification for the assumption.} 
% \plt{If it helps with the organization and understandability, the assumptions
% can be presented as sub sections.  The following sub-sections are options:
% background theory assumptions, helper theory assumptions, generic theory
% assumptions, problem specific assumptions, and rationale assumptions}

This section simplifies the original problem and helps in developing the
theoretical model by filling in the missing information for the physical system.
The numbers given in the square brackets refer to the theoretical model [TM],
general definition [GD], data definition [DD], instance model [IM], or likely
change [LC], in which the respective assumption is used.

\begin{itemize}

\item[A\refstepcounter{assumpnum}\theassumpnum \label{dna-sequences-only}:]
DNA-sequences-only: The only type of genetic sequences that will be considered are DNA sequences
\item[A\refstepcounter{assumpnum}\theassumpnum \label{two-sequences-only}:]
Two-sequences-only: The only valid number of genetic sequences to be compared is two
  % \plt{Short description of each assumption.  Each assumption
  %   should have a meaningful label.  Use cross-references to identify the
  %   appropriate traceability to TM, GD, DD etc., using commands like dref, ddref
  %   etc.  Each assumption should be atomic - that is, there should not be an
  %   explicit (or implicit) ``and'' in the text of an assumption.}

\end{itemize}

\subsubsection{Theoretical Models}\label{sec_theoretical}

% \plt{Theoretical models are sets of abstract mathematical equations or axioms
%   for solving the problem described in Section ``Physical System Description''
%   (Section~\ref{sec_phySystDescrip}). Examples of theoretical models are
%   physical laws, constitutive equations, relevant conversion factors, etc.}

% \plt{Optionally the theory section could be divided into subsections to provide
% more structure and improve understandability and reusability.  Potential
% subsections include the following: Context theories, background theories, helper
% theories, generic theories, problem specific theories, final theories and
% rationale theories.}

This section focuses on the general equations and laws that \progname{} is based
on.  
% \plt{Modify the examples below for your problem, and add additional models
%   as appropriate.}

~\newline

\noindent
\deftheory
% #2 refname of theory
{TM:DPO}
% #3 label
{Dynamic Programming Optimization}
% #4 equation
% {
%   % $F_{ij} = \max(F_{i-1,j-1} + S(SEQ_{A_i}, SEQ_{B_j}), F_{i,j-1} + g, F_{i-1,j} + g)$
%   $F_{i,j} = \max(F_{i-1,j-1} + S(A_{i}, B_{j}), F_{i,j-1} + d, F_{i-1,j} + d)$
% }
{
  % $F_{i,j} = \max(F_{i-1,j-1} + S(A_{i}, B_{j}), F_{i,j-1} + d, F_{i-1,j} + d)$
  $F_{ij} = \max(F_{i-1,j-1} + S(SEQ_{A_i}, SEQ_{B_j}), F_{i,j-1} + g, F_{i-1,j} + g)$
}
% #5 description
{
  % The above equation gives the optimal alignment score between two genetic sequences $A_{i}$ and $B$,
  % where $F_{i,j}$ is the alignment score at position $i$ in genetic sequence $A$ and position $j$ in 
  % genetic sequence $B$, $d$ is the gap penalty associated with a deletion or insertion, and $S$ is the substitution matrix.

  The above equation gives the optimal alignment score between two genetic sequences $SEQ_{A_i}$ and $SEQ_{B_j}$,
  given in base pair (bp) units, where $F_{i,j}$ is the alignment score at position $i$ in genetic sequence $SEQ_A$ 
  and position $j$ in genetic sequence $SEQ_B$ (in qa units), $g$ is the gap penalty associated with a deletion 
  or insertion (given in qa units), and $S$ is the substitution matrix (given in qa units).
  % The above equation gives the conservation of energy for transient heat transfer in a material
  % of specific heat capacity $C$ (\si{\joule\per\kilogram\per\celsius}) and density $\rho$ 
  % (\si{\kilogram\per\cubic\metre}), where $\bf q$ is the thermal flux vector (\si{\watt\per\square\metre}),
  % $g$ is the volumetric heat generation
  % (\si{\watt\per\cubic\metre}), $T$ is the temperature
  % (\si{\celsius}),  $t$ is time (\si{\second}), and $\nabla$ is
  % the gradient operator.  For this equation to apply, other forms
  % of energy, such as mechanical energy, are assumed to be negligible in the
  % system (\aref{A_OnlyThermalEnergy}).  In general, the material properties ($\rho$ and $C$) depend on temperature.
}
% #6 Notes
{
None.
}
% #7 Source
{
  \href{https://en.wikipedia.org/wiki/Needleman-Wunsch_algorithm}{Needleman-Wunsch Algorithm}, \cite{NEEDLEMAN1970443}
  % \url{https://en.wikipedia.org/wiki/Needleman-Wunsch_algorithm}
}
% #8 Referenced by
{
  % \dref{ROCT}
  --
}
% #9 Preconditions
{
None
}
% #1 derivation - not applicable by default
{}

% \plt{``Ref.\ By'' is used repeatedly with the different types of information.
%   This stands for Referenced By.  It means that the models, definitions and
%   assumptions listed reference the current model, definition or assumption.
%   This information is given for traceability.  Ref. By provides a pointer in the
%   opposite direction to what we commonly do.  You still need to have a reference
%   in the other direction pointing to the current model, definition or
%   assumption.  As an example, if TM1 is referenced by GD2, that means that GD2 will
%   explicitly include a reference to TM1.}

~\newline

\subsubsection{General Definitions}\label{sec_gendef}

% \plt{General Definitions (GDs) are a refinement of one or more TMs, and/or of
%   other GDs.  The GDs are less abstract than the TMs.  Generally the reduction
%   in abstraction is possible through invoking (using/referencing) Assumptions.
%   For instance, the TM could be Newton's Law of Cooling stated abstracting.  The
%   GD could take the general law and apply it to get a 1D equation.}

This section collects the laws and equations that will be used in building the
instance models.

% \plt{Some projects may not have any content for this section, but the section
%   heading should be kept.}  \plt{Modify the examples below for your problem, and
%   add additional definitions as appropriate.}

% ~\newline

% \noindent
% \begin{minipage}{\textwidth}
% \renewcommand*{\arraystretch}{1.5}
% \begin{tabular}{| p{\colAwidth} | p{\colBwidth}|}
% \hline
% \rowcolor[gray]{0.9}
% Number& GD\refstepcounter{defnum}\thedefnum \label{NL}\\
% \hline
% Label &\bf Needleman-Wunsch algorithm \\
% \hline
% % Units&$MLt^{-3}T^0$\\
% % \hline
% SI Units& {} \\
% \hline
% Equation&$F_{ij} = \max(F_{i-1,j-1} + S(SEQ_{A_i}, SEQ_{B_j}), F_{i,j-1} + g, F_{i-1,j} + g)$  \\
% \hline
% Description & 
% The above equation gives the optimal alignment score between two genetic sequences $SEQ_{A,i}$ and $SEQ_{B,j}$
% given in base pair (bp) units, where $F_{i,j}$ is the alignment score at position $i$ in genetic sequence $SEQ_A$ 
% and position $j$ in genetic sequence $SEQ_B$ (in qa units), $g$ is the gap penalty associated with a deletion 
% or insertion (given in qa units), and $S$ is the substitution matrix (given in qa units).
% \\
% & $q(t)$ is the thermal flux (\si{\watt\per\square\metre}).\\
% & $h$ is the heat transfer coefficient, assumed independent of $T$ (\aref{A_hcoeff})
% 	(\si{\watt\per\square\metre\per\celsius}).\\
% &$\Delta T(t)$= $T(t) - T_{\text{env}}(t)$ is the time-dependent thermal gradient
% between the environment and the object (\si{\celsius}).
% \\
% \hline
%   Source & Citation here \\
%   \hline
%   Ref.\ By & \ddref{FluxCoil}, \ddref{FluxPCM}\\
%   \hline
% \end{tabular}
% \end{minipage}\\

% \subsubsection*{Detailed derivation of simplified rate of change of temperature}

% No detailed derivation necessary to describe the recursive nature of the algorithm
% \plt{This may be necessary when the necessary information does not fit in the
%   description field.}
% \plt{Derivations are important for justifying a given GD.  You want it to be
%   clear where the equation came from.}

\subsubsection{Data Definitions}\label{sec_datadef}

% \plt{The Data Definitions are definitions of symbols and equations that are
%   given for the problem.  They are not derived; they are simply used by other
%   models.  For instance, if a problem depends on density, there may be a data
%   definition for the equation defining density.  The DDs are given information
%   that you can use in your other modules.}

% \plt{All Data Definitions should be used (referenced) by at least one other
%   model.}

This section collects and defines all the data needed to build the instance
models. The dimension of each quantity is also given.  
% \plt{Modify the examples
%   below for your problem, and add additional definitions as appropriate.}

~\newline

% %%%%%%%%%%%%%%%%%%%%%%%%%%%%%%%%%%%%%%

\noindent
\begin{minipage}{\textwidth}
\renewcommand*{\arraystretch}{1.5}
\begin{tabular}{| p{\colAwidth} | p{\colBwidth}|}
\hline
\rowcolor[gray]{0.9}
Number& DD\refstepcounter{datadefnum}\thedatadefnum \label{DD_comparative_alginment_matrix}\\
\hline
Label& \bf Comparative alignment matrix\\
\hline
Symbol &$F$\\
\hline
% Units& $Mt^{-3}$\\
% \hline
  Units & qa \\
  \hline
  Equation&$F_{ij} = \max(F_{i-1,j-1} + S(SEQ_{A_i}, SEQ_{B_j}), F_{i,j-1} + g, F_{i-1,j} + g)$\\
  \hline
  Description & 
                Comparative matrix of the alignment between two sequences, encoding the positional 
                quality of each possible combination of the base pairs that conform such alignment.
  \\
  \hline
  Sources& -- \\
  \hline
  Ref.\ By & \iref{IM_needleman_wunsch}\\
  \hline
\end{tabular}
\end{minipage}\\


\noindent
\begin{minipage}{\textwidth}
\renewcommand*{\arraystretch}{1.5}
\begin{tabular}{| p{\colAwidth} | p{\colBwidth}|}
\hline
\rowcolor[gray]{0.9}
Number& DD\refstepcounter{datadefnum}\thedatadefnum \label{DD_substitution_matrices}\\
\hline
Label& \bf Set of substitution matrices\\
\hline
Symbol &$\mathbb{S}$\\
\hline
% Units& $Mt^{-3}$\\
% \hline
  Units & -- \\
  \hline
  Equation&$S_k \in \{S_1, S_2, ..., S_n\}$\\
  \hline
  Description & 
                Set of substitution matrices that will be used
                to calculate the alignment quality between two genetic sequences.
  \\
  \hline
  Sources& -- \\
  \hline
  Ref.\ By & \iref{IM_needleman_wunsch}\\
  \hline
\end{tabular}
\end{minipage}\\




% \subsubsection{Data Types}\label{sec_datatypes}

% \plt{This section is optional.  In many scientific computing programs it isn't
%   necessary, since the inputs and outpus are straightforward types, like reals,
%   integers, and sequences of reals and integers.  However, for some problems it
%   is very helpful to capture the type information.}

% \plt{The data types are not derived; they are simply stated and used by other
%   models.}

% \plt{All data types must be used by at least one of the models.}

% \plt{For the mathematical notation for expressing types, the recommendation is
%   to use the notation of~\citet{HoffmanAndStrooper1995}.}

% This section collects and defines all the data types needed to document the
% models. \plt{Modify the examples below for your problem, and add additional
%   definitions as appropriate.}

% ~\newline

% \noindent
% \begin{minipage}{\textwidth}
% \renewcommand*{\arraystretch}{1.5}
% \begin{tabular}{| p{\colAwidth} | p{\colBwidth}|}
%   \hline
%   \rowcolor[gray]{0.9}
%   Type Name & Name for Type\\
%   \hline
%   Type Def & mathematical definition of the type\\
%   \hline
%   Description & description here
%   \\
%   \hline
%   Sources & Citation here, if the type is borrowed from another source\\
%   \hline
% \end{tabular}
% \end{minipage}\\

\subsubsection{Instance Models} \label{sec_instance}    

% \plt{The motivation for this section is to reduce the problem defined in
%   ``Physical System Description'' (Section~\ref{sec_phySystDescrip}) to one
%   expressed in mathematical terms. The IMs are built by refining the TMs and/or
%   GDs.  This section should remain abstract.  The SRS should specify the
%   requirements without considering the implementation.}

This section transforms the problem defined in Section~\ref{Sec_pd} into 
one which is expressed in mathematical terms. It uses concrete symbols defined 
in Section~\ref{sec_datadef} to replace the abstract symbols in the models 
identified in Sections~\ref{sec_theoretical} and~\ref{sec_gendef}.


% The goals \plt{reference your goals} are solved by \plt{reference your instance
%   models}.  

% \plt{other details, with cross-references where appropriate.}
% \plt{Modify the examples below for your problem, and add additional models as
%   appropriate.}

~\newline

%Instance Model 1
The goal \gsref{generate_alignment} is met by \iref{IM_needleman_wunsch}. 

\noindent
\begin{minipage}{\textwidth}
\renewcommand*{\arraystretch}{1.5}
\begin{tabular}{| p{\colAwidth} | p{\colBwidth}|}
  \hline
  \rowcolor[gray]{0.9}
  Number& IM\refstepcounter{instnum}\theinstnum \label{IM_needleman_wunsch}\\
  \hline
  Label& $O$\\
  \hline
  Input&$SEQ_A$,$SEQ_B$,$S_k$,$g$\\
  \hline
  Output&$O_{AB}$\\
  \hline
  Description& $O_{AB} = \forall S_k \in \mathbb{S} : F_{i,j}^k = \max(F_{i-1,j-1}^k + S_k(SEQ_A, SEQ_B), F_{i,j-1}^k + g, F_{i-1,j}^k + g)$\\
  &$SEQ_A$ and $SEQ_B$ are biological genetic sequences with bp units\\
  &$S_k$ is a substitution matrix element of $\mathbb{S}$\\
  &$g$ is the gap penalty given in qa units\\
  &$F$ is comparison matrix between sequences, with each cell given in qa units\\
  \hline
  Sources& -- \\
  \hline
  Ref.\ By & --\\
  \hline
\end{tabular}
\end{minipage}\\

%~\newline

% \subsubsection*{Derivation of ...}

% \plt{The derivation shows how the IM is derived from the TMs/GDs.  In cases
%   where the derivation cannot be described under the Description field, it will
%   be necessary to include this subsection.}

\subsubsection{Input Data Constraints} \label{sec_DataConstraints}    

Table~\ref{TblInputVar} shows the data constraints on the input output
variables.  The column for physical constraints gives the physical limitations
on the range of values that can be taken by the variable.  The column for
software constraints restricts the range of inputs to reasonable values.  The
software constraints will be helpful in the design stage for picking suitable
algorithms.  The constraints are conservative, to give the user of the model the
flexibility to experiment with unusual situations.  The column of typical values
is intended to provide a feel for a common scenario.  The uncertainty column
provides an estimate of the confidence with which the physical quantities can be
measured.  This information would be part of the input if one were performing an
uncertainty quantification exercise.

% The specification parameters in Table~\ref{TblInputVar} are listed in
% Table~\ref{TblSpecParams}.

\begin{table}[!h]
  \caption{Input Variables} \label{TblInputVar}
  \renewcommand{\arraystretch}{1.2}
\noindent \begin{longtable*}{l l l l c} 
  \toprule
  \textbf{Var} & \textbf{Physical Constraints} & \textbf{Software Constraints} &
                             \textbf{Typical Value} & \textbf{Uncertainty}\\
  \midrule 
  $SEQ_A$ & $|{seq_B}| \geq 1$ & $|seq_A| \approx |seq_B|$ & 1 kb & 40\%
  \\
  $SEQ_B$ & $|{seq_A}| \geq 1$ & $|seq_B| \approx |seq_A|$ & 1 kb & 40\%
  \\
  $S_k$ & $S \in \mathbb{R}^{n \times n}, n \geq 4$ & $S \in \mathbb{R}^{n \times n}, n \geq 0$ & $S \in \mathbb{R}^{4 \times 4}$& 0\%
  \\
  $F$ & $F \in \sum|seq_i| \times |seq_j|$ & $|seq_i|, |seq_j| \geq 1$ & $\approx 1 kb^2$ & 20\%
  \\
  $g$ & $g \in \mathbb{R}_{\leq 0}$ & -- & $-2$ & 10\%
  \\
  $O_{AB}$ & $O_{AB} \in \mathbb{R}^{m \times n}, m,n \geq 1$ & -- & *$\vec{v} = [0,-2,-12]$ & 0\%
  \\
  \bottomrule
\end{longtable*}
\end{table}

\noindent 
\begin{description}
\item[(*)] The vector $O_{AB}$ is presented as a named collection of scores
\end{description}

% \begin{table}[!h]
% \caption{Specification Parameter Values} \label{TblSpecParams}
% \renewcommand{\arraystretch}{1.2}
% \noindent \begin{longtable*}{l l} 
%   \toprule
%   \textbf{Var} & \textbf{Value} \\
%   \midrule 
%   $L_\text{min}$ & 0.1 \si{\metre}\\
%   \bottomrule
% \end{longtable*}
% \end{table}

% \subsubsection{Properties of a Correct Solution} \label{sec_CorrectSolution}

% \noindent
% A correct solution must exhibit \plt{fill in the details}.  \plt{These
%   properties are in addition to the stated requirements.  There is no need to
%   repeat the requirements here.  These additional properties may not exist for
%   every problem.  Examples include conservation laws (like conservation of
%   energy or mass) and known constraints on outputs, which are usually summarized
%   in tabular form.  A sample table is shown in Table~\ref{TblOutputVar}}

% \begin{table}[!h]
% \caption{Output Variables} \label{TblOutputVar}
% \renewcommand{\arraystretch}{1.2}
% \noindent \begin{longtable*}{l l} 
%   \toprule
%   \textbf{Var} & \textbf{Physical Constraints} \\
%   \midrule 
%   $T_W$ & $T_\text{init} \leq T_W \leq T_C$ (by~\aref{A_charge})
%   \\
%   \bottomrule
% \end{longtable*}
% \end{table}

% \plt{This section is not for test cases or techniques for verification and
%   validation.  Those topics will be addressed in the Verification and Validation
%   plan.}

\section{Requirements}

% \plt{The requirements refine the goal statement.  They will make heavy use of
%   references to the instance models.}

This section provides the functional requirements, the business tasks that the
software is expected to complete, and the nonfunctional requirements, the
qualities that the software is expected to exhibit.

\subsection{Functional Requirements}

\noindent \begin{itemize}

\item[R\refstepcounter{reqnum}\thereqnum \label{R_Inputs}:] Input $SEQ_A$, $SEQ_B$ as 
strings of base pair units (bp), substitution matrix $S \in \mathbb{R}^{n \times n}$, and gap penalty $g \in \mathbb{R}_{<0}$.

% \plt{Requirements
%     for the inputs that are supplied by the user.  This information has to be
%     explicit.}

\item[R\refstepcounter{reqnum}\thereqnum \label{R_OutputInputs}:] Use the inputs 
stated in \iref{R_Inputs} to build a comparative matrix $F^k$ for 
each substitution matrix $S_k$ in $\mathbb{S}$.

% \plt{It isn't
%   always required, but often echoing the inputs as part of the output is a
%   good idea.}

\item[R\refstepcounter{reqnum}\thereqnum \label{R_Calculate}:] Calculate optimal 
alignment scores using dynamic programming recursion \iref{IM_needleman_wunsch}.

\item[R\refstepcounter{reqnum}\thereqnum \label{R_VerifyOutput}:] Verify that:
\begin{itemize}
\item Input sequences contain only valid nucleotides (A,T,C,G)
\item Sequences meet minimum length requirement $|seq_i|, |seq_j| \geq 1$
\item Gap penalty is negative $g < 0$
\item Substitution matrices are square $n \times n$
\end{itemize}

\item[R\refstepcounter{reqnum}\thereqnum \label{R_Output}:] Output:
\begin{itemize}
\item Aligned sequences with gap insertions
\item Alignment scores for each $S_k$
\item Ranking of substitution matrices by alignment quality
\end{itemize}

\end{itemize}

% \plt{Every IM should map to at least one requirement, but not every requirement
%   has to map to a corresponding IM.}

\subsection{Nonfunctional Requirements}

% \plt{List your nonfunctional requirements.  You may consider using a fit
%   criterion to make them verifiable.}
% \plt{The goal is for the nonfunctional requirements to be unambiguous, abstract
%   and verifiable.  This isn't easy to show succinctly, so a good strategy may be
% to give a ``high level'' view of the requirement, but allow for the details to
% be covered in the Verification and Validation document.}
% \plt{An absolute requirement on a quality of the system is rarely needed.  For
%   instance, an accuracy of 0.0101 \% is likely fine, even if the requirement is
%   for 0.01 \% accuracy.  Therefore, the emphasis will often be more on
%   describing now well the quality is achieved, through experimentation, and
%   possibly theory, rather than meeting some bar that was defined a priori.}
% \plt{You do not need an entry for correctness in your NFRs.  The purpose of the
%   SRS is to record the requirements that need to be satisfied for correctness.
%   Any statement of correctness would just be redundant. Rather than discuss
%   correctness, you can characterize how far away from the correct (true)
%   solution you are allowed to be.  This is discussed under accuracy.}

\noindent \begin{itemize}

  \item[NFR\refstepcounter{nfrnum}\thenfrnum \label{NFR_Accuracy}:]
  \textbf{Accuracy} The alignment quality scores produced by \progname{} shall meet the 
  precision requirements needed for comparative biology research. 
  
  \item[NFR\refstepcounter{nfrnum}\thenfrnum \label{NFR_Usability}:] \textbf{Usability}
  Users with knowledge of genetics and comparative biology, as described in Section~\ref{SecUserCharacteristics}, 
  should be able to successfully use the software with minimal training. 
  The interface shall accept standard sequence formats and provide clear visualization of alignments.
  
  \item[NFR\refstepcounter{nfrnum}\thenfrnum \label{NFR_Maintainability}:]
  \textbf{Maintainability} The effort required to modify or extend \progname{} 
  with new substitution matrices (e.g. protein matrices) should be less than 5\% 
  of the original development time, and no more than 40\% of the original development time 
  for new alignment algorithms (e.g. heuristics).
  
  \item[NFR\refstepcounter{nfrnum}\thenfrnum \label{NFR_Portability}:]
  \textbf{Portability} \progname{} shall run on Linux, Windows 10+, and MacOS 13+ operating systems.
  
  \item[NFR\refstepcounter{nfrnum}\thenfrnum \label{NFR_Performance}:]
  \textbf{Performance} \progname{} shall complete alignment calculations for sequences of length n in O(n²) time complexity.

\end{itemize}

\subsection{Rationale}

The rationale behind the assumption \aref{dna-sequences-only} relies on the unique 
benchmarking properties of the set $\mathbb{S}$ that contains only DNA substitution matrices.
This improves on the user experience and improves the modularity in the software, enhancing 
maintainability and portability, which are key nonfunctional requirements.

The second rationale that justifies assumption \aref{two-sequences-only} is the
nature of the Needleman-Wunsch algorithm, which guarantees optimal alignment in 2D matrices.
% \plt{Provide a rationale for the decisions made in the documentation.  Rationale
% should be provided for scope decisions, modelling decisions, assumptions and
% typical values.}

\section{Likely Changes}    

\noindent \begin{itemize}

\item[LC\refstepcounter{lcnum}\thelcnum\label{LC_expand_matrices}:]
The software may be extended to include protein sequences, which will require
expanding the set of substitution matrices $\mathbb{S}$ to include protein matrices.

\end{itemize}

\section{Unlikely Changes}    

\noindent \begin{itemize}

\item[LC\refstepcounter{lcnum}\thelcnum\label{LC_keep_optimization_global}:] 
The dimensionality of matrix $F$ shall remain 2D, as the Needleman-Wunsch algorithm
is designed to optimize global alignment scores.

\end{itemize}

\section{Traceability Matrices and Graphs}

The purpose of the traceability matrices is to provide easy references on what
has to be additionally modified if a certain component is changed.  Every time a
component is changed, the items in the column of that component that are marked
with an ``X'' may have to be modified as well.  Table~\ref{Table:trace} shows the
dependencies of theoretical models, general definitions, data definitions, and
instance models with each other. Table~\ref{Table:R_trace} shows the
dependencies of instance models, requirements, and data constraints on each
other. Table~\ref{Table:A_trace} shows the dependencies of theoretical models,
general definitions, data definitions, instance models, and likely changes on
the assumptions.

% \plt{You will have to modify these tables for your problem.}

% \plt{The traceability matrix is not generally symmetric.  If GD1 uses A1, that
%   means that GD1's derivation or presentation requires invocation of A1.  A1
%   does not use GD1.  A1 is ``used by'' GD1.}

% \plt{The traceability matrix is challenging to maintain manually.  Please do
%   your best.  In the future tools (like Drasil) will make this much easier.}

% \afterpage{
% \begin{landscape}
\begin{table}[h!]
\centering
% \begin{tabular}{|c|c|c|c|c|c|c|c|c|c|c|c|c|c|c|c|c|c|c|c|}
% \hline
% 	& \aref{A_OnlyThermalEnergy}& \aref{A_hcoeff}& \aref{A_mixed}& \aref{A_tpcm}& \aref{A_const_density}& \aref{A_const_C}& \aref{A_Newt_coil}& \aref{A_tcoil}& \aref{A_tlcoil}& \aref{A_Newt_pcm}& \aref{A_charge}& \aref{A_InitTemp}& \aref{A_OpRangePCM}& \aref{A_OpRange}& \aref{A_htank}& \aref{A_int_heat}& \aref{A_vpcm}& \aref{A_PCM_state}& \aref{A_Pressure} \\
% \hline
% \tref{T_COE}        & X& & & & & & & & & & & & & & & & & & \\ \hline
% \tref{T_SHE}        & & & & & & & & & & & & & & & & & & & \\ \hline
% \tref{T_LHE}        & & & & & & & & & & & & & & & & & & & \\ \hline
% \dref{NL}           & & X& & & & & & & & & & & & & & & & & \\ \hline
% \dref{ROCT}         & & & X& X& X& X& & & & & & & & & & & & & \\ \hline
% \ddref{FluxCoil}    & & & & & & & X& X& X& & & & & & & & & & \\ \hline
% \ddref{FluxPCM}     & & & X& X& & & & & & X& & & & & & & & & \\ \hline
% \ddref{D_HOF}       & & & & & & & & & & & & & & & & & & & \\ \hline
% \ddref{D_MF}        & & & & & & & & & & & & & & & & & & & \\ \hline
% \iref{ewat}         & & & & & & & & & & & X& X& & X& X& X& & & X \\ \hline
% \iref{epcm}         & & & & & & & & & & & & X& X& & & X& X& X& \\ \hline
% \iref{I_HWAT}       & & & & & & & & & & & & & & X& & & & & X \\ \hline
% \iref{I_HPCM}       & & & & & & & & & & & & & X& & & & & X & \\ \hline
% \lcref{LC_tpcm}     & & & & X& & & & & & & & & & & & & & & \\ \hline
% \lcref{LC_tcoil}    & & & & & & & & X& & & & & & & & & & & \\ \hline
% \lcref{LC_tlcoil}   & & & & & & & & & X& & & & & & & & & & \\ \hline
% \lcref{LC_charge}   & & & & & & & & & & & X& & & & & & & & \\ \hline
% \lcref{LC_InitTemp} & & & & & & & & & & & & X& & & & & & & \\ \hline
% \lcref{LC_htank}    & & & & & & & & & & & & & & & X& & & & \\
% \hline
% \end{tabular}
\begin{tabular}{|c|c|c|c|c|c|}
\hline
  & \aref{dna-sequences-only}& \aref{two-sequences-only}& \tref{TM:DPO}& \ddref{DD_comparative_alginment_matrix}& \iref{IM_needleman_wunsch}\\
\hline
\aref{dna-sequences-only} & --&  &  &  &  \\ \hline
\aref{two-sequences-only} &  & --&  &  &  \\ \hline
\tref{TM:DPO} & X & X & --&  &  \\ \hline
\ddref{DD_comparative_alginment_matrix} & X & X &  & --&  \\ \hline
\iref{IM_needleman_wunsch} & X & X & X &  & --\\
\hline
\end{tabular}


\caption{Traceability Matrix Showing the Connections Between Assumptions and Other Items}
\label{Table:A_trace}
\end{table}
% \end{landscape}
% }

%%%%%%%%%%%%%%%%%%%%%%%%%%%%%%%%%%%%%

\begin{table}[h!]
\centering
\begin{tabular}{|c|c|c|c|c|}
\hline        
  & \tref{TM:DPO} & \ddref{DD_comparative_alginment_matrix} & \ddref{DD_substitution_matrices} & \iref{IM_needleman_wunsch} \\ \hline
\tref{TM:DPO} & -- & X &   & X \\ \hline
\ddref{DD_comparative_alginment_matrix} &   & -- &   &   \\ \hline
\ddref{DD_substitution_matrices} &   &   & -- &   \\ \hline
\iref{IM_needleman_wunsch} & X & X &   & -- \\ \hline
\end{tabular}
\caption{Traceability Matrix Showing the Connections Between Items of Different Sections}
\label{Table:trace}
\end{table}

%%%%%%%%%%%%%%%%%%%%%%%%%%%%%%%%%%%%%

% \begin{tabular}{|c|c|c|c|c|}
% \hline
%   & \iref{IM_needleman_wunsch} & \ref{sec_DataConstraints} & \rref{R_Inputs} & \rref{R_Calculate} \\ \hline
% \iref{IM_needleman_wunsch} & -- & X & X & X \\  \hline
% \rref{R_Inputs} & X & X & -- & X \\  \hline
% \rref{R_OutputInputs} & X & X & X & X \\  \hline
% \rref{R_Calculate} & X & X & X & -- \\  \hline
% \rref{R_VerifyOutput} & X & X & X & X \\  \hline
% \rref{R_Output} & X & X & X & X \\ \hline
% \end{tabular}


\begin{table}[h!]
\centering
\begin{tabular}{|c|c|c|c|c|c|c|}
\hline
  & \iref{IM_needleman_wunsch} & \rref{R_Inputs} & \rref{R_OutputInputs} & \rref{R_Calculate} & \rref{R_VerifyOutput} & \rref{R_Output} \\ \hline
\iref{IM_needleman_wunsch} & -- & X & & X & & X \\ \hline
\rref{R_Inputs} & X & -- & X & & & \\ \hline
\rref{R_OutputInputs} & & X & -- & & & \\ \hline
\rref{R_Calculate} & X & & & -- & & X \\ \hline
\rref{R_VerifyOutput} & & X & & X & -- & \\ \hline
\rref{R_Output} & X & & & X & & -- \\ \hline
\end{tabular}
\caption{Traceability Matrix Showing the Connections Between Requirements and Instance Models}
\label{Table:R_trace}
\end{table}

The purpose of the traceability graphs is also to provide easy references on
what has to be additionally modified if a certain component is changed.  The
arrows in the graphs represent dependencies. The component at the tail of an
arrow is depended on by the component at the head of that arrow. Therefore, if a
component is changed, the components that it points to should also be
changed. Figure~\ref{Fig_ATrace} shows the dependencies of theoretical models,
general definitions, data definitions, instance models, likely changes, and
assumptions on each other. Figure~\ref{Fig_RTrace} shows the dependencies of
instance models, requirements, and data constraints on each other.

\begin{figure}[h!]
	\begin{center}
		%\rotatebox{-90}
		{
			\includegraphics[width=\textwidth]{sublimat_atrace.png}
		}
		\caption{\label{Fig_ATrace} Traceability Matrix Showing the Connections Between Items of Different Sections}
	\end{center}
\end{figure}


\begin{figure}[h!]
	\begin{center}
		%\rotatebox{-90}
		{
			\includegraphics[width=0.7\textwidth]{sublimat_rtrace.png}
		}
		\caption{\label{Fig_RTrace} Traceability Matrix Showing the Connections Between Requirements, Instance Models, and Data Constraints}
	\end{center}
\end{figure}

% \section{Development Plan}

% \plt{This section is optional.  It is used to explain the plan for developing
%   the software.  In particular, this section gives a list of the order in which
%   the requirements will be implemented.  In the context of a course  this is
%   where you can indicate which requirements will be implemented as part of the
%   course, and which will be ``faked'' as future work.  This section can be
%   organized as a prioritized list of requirements, or it could should the
%   requirements that will be implemented for ``phase 1'', ``phase 2'', etc.}

\section{Values of Auxiliary Constants}

% \plt{Show the values of the symbolic parameters introduced in the report.}

% \plt{The definition of the requirements will likely call for SYMBOLIC\_CONSTANTS.
% Their values are defined in this section for easy maintenance.}

% \plt{The value of FRACTION, for the Maintainability NFR would be given here.}
\begin{table}[h!]
\centering
\begin{tabular}{|l|l|c|c|}
\hline
\textbf{Symbol} & \textbf{Description} & \textbf{Value} & \textbf{Unit} \\
\hline
$BS$ & Baseline substitution matrix $s \in \mathbb{S}$ & $\begin{bmatrix}
0 & -3 & -1 & -3 \\
-3 & 0 & -3 & -1 \\
-1 & -3 & 0 & -3 \\
-3 & -1 & -3 & 0
\end{bmatrix}$ & qa \\
\hline
$JC$ & Jukes Cantor substitution matrix $s \in \mathbb{S}$ & $\begin{bmatrix}
1.0 & -\frac{1}{3} & -\frac{1}{3} & -\frac{1}{3} \\
-\frac{1}{3} & 1.0 & -\frac{1}{3} & -\frac{1}{3} \\
-\frac{1}{3} & -\frac{1}{3} & 1.0 & -\frac{1}{3} \\
-\frac{1}{3} & -\frac{1}{3} & -\frac{1}{3} & 1.0
\end{bmatrix}$ & qa \\
\hline
$K80$ & Kimura 1980 substitution matrix $s \in \mathbb{S}$ & $\begin{bmatrix}
1.0 & -2.0 & -1.0 & -2.0 \\
-2.0 & 1.0 & -2.0 & -1.0 \\
-1.0 & -2.0 & 1.0 & -2.0 \\
-2.0 & -1.0 & -2.0 & 1.0
\end{bmatrix}$ & qa \\
\hline
$HKY85$ & Hasegawa-Kishino-Yano 1985 matrix $s \in \mathbb{S}$ & $\begin{bmatrix}
1.0 & -2.5 & -1.0 & -2.5 \\
-2.5 & 1.0 & -2.5 & -1.0 \\
-1.0 & -2.5 & 1.0 & -2.5 \\
-2.5 & -1.0 & -2.5 & 1.0
\end{bmatrix}$ & qa \\
\hline
$TN93$ & Tamura-Nei 1993 substitution matrix $s \in \mathbb{S}$ & $\begin{bmatrix}
1.0 & -2.5 & -1.0 & -2.5 \\
-2.5 & 1.0 & -2.5 & -1.5 \\
-1.0 & -2.5 & 1.0 & -2.5 \\
-2.5 & -1.5 & -2.5 & 1.0
\end{bmatrix}$ & qa \\
\hline
\end{tabular}
\caption{Values of Auxiliary Constants}
\label{table:aux_constants}
\end{table}

\newpage

\bibliographystyle {plainnat}
\bibliography {../../refs/References}

% \newpage

% \noindent \plt{The following is not part of the template, just some things to consider
%   when filing in the template.}

% \noindent \plt{Grammar, flow and \LaTeX advice:
% \begin{itemize}
% \item For Mac users \texttt{*.DS\_Store} should be in \texttt{.gitignore}
% \item \LaTeX{} and formatting rules
% \begin{itemize}
% \item Variables are italic, everything else not, includes subscripts (link to
%   document)
% \begin{itemize}
% \item \href{https://physics.nist.gov/cuu/pdf/typefaces.pdf}{Conventions}
% \item Watch out for implied multiplication
% \end{itemize}
% \item Use BibTeX
% \item Use cross-referencing
% \end{itemize}
% \item Grammar and writing rules
% \begin{itemize}
% \item Acronyms expanded on first usage (not just in table of acronyms)
% \item ``In order to'' should be ``to''
% \end{itemize}
% \end{itemize}}

% \noindent \plt{Advice on using the template:
% \begin{itemize}
% \item Difference between physical and software constraints
% \item Properties of a correct solution means \emph{additional} properties, not
%   a restating of the requirements (may be ``not applicable'' for your problem).
%   If you have a table of output constraints, then these are properties of a
%   correct solution.
% \item Assumptions have to be invoked somewhere
% \item ``Referenced by'' implies that there is an explicit reference
% \item Think of traceability matrix, list of assumption invocations and list of
%   reference by fields as automatically generatable
% \item If you say the format of the output (plot, table etc), then your
%   requirement could be more abstract
% \end{itemize}
% }

% \newpage{}
% \section*{Appendix --- Reflection}

% \wss{Not required for CAS 741}

% The information in this section will be used to evaluate the team members on the
% graduate attribute of Lifelong Learning.  

% \input{../Reflection.tex}

% \input{../SRS_Reflection.tex}

\end{document}