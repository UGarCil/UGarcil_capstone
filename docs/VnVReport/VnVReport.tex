\documentclass[12pt, titlepage]{article}

% Preamble: all package declarations go here
\usepackage{tabularx}
\usepackage{graphicx}
\usepackage{array} 
\usepackage{booktabs}
\usepackage{hyperref}
\usepackage{amssymb}
\usepackage[round]{natbib}

\hypersetup{
    colorlinks,
    citecolor=black,
    filecolor=black,
    linkcolor=red,
    urlcolor=blue
}

% Input external macros or settings
\input{../Comments}
%% Common Parts

\newcommand{\progname}{SubLiMat} % PUT YOUR PROGRAM NAME HERE
\newcommand{\authname}{Uriel Garcilazo Cruz} % AUTHOR NAMES                  

\usepackage{hyperref}
    \hypersetup{colorlinks=true, linkcolor=blue, citecolor=blue, filecolor=blue,
                urlcolor=blue, unicode=false}
    \urlstyle{same}
                                

\begin{document}

\title{Verification and Validation Report: \progname} 
\author{\authname}
\date{\today}
	
\maketitle

\pagenumbering{roman}

\section{Revision History}

\begin{tabularx}{\textwidth}{p{3cm}p{2cm}X}
\toprule {\bf Date} & {\bf Version} & {\bf Notes}\\
\midrule
Apr. 10th & 1.0 & First version\\
\bottomrule
\end{tabularx}

\newpage

\section{Symbols, Abbreviations and Acronyms}

\renewcommand{\arraystretch}{1.2}
\begin{tabular}{l l} 
  \toprule		
  \textbf{Symbol} & \textbf{Description}\\
  \midrule 
  TC & Test Case\\
  SRS & Software Requirements Specification\\
  VnV & Verification and Validation\\
  \bottomrule
\end{tabular}

\newpage

\tableofcontents
\listoftables
\listoffigures

\newpage

\pagenumbering{arabic}

This document presents the verification and validation results for Substitution Matrix Benchmarking with Pairwise Sequence Alignment (\progname). The report demonstrates how testing activities confirmed the software meets all specified requirements.

\section{Functional Requirements Evaluation}

\begin{table}[h]
\centering
\caption{Functional Requirements Test Results}
\begin{tabularx}{\textwidth}{lXc}
\toprule
\textbf{Requirement} & \textbf{Test Cases} & \textbf{Result} \\
\midrule
R1: Valid Sequence Input & TC-SubLiMat-1-1 to TC-SubLiMat-1-8 & \checkmark \\
R2: Matrix Construction & TC-SubLiMat-2-1 to TC-SubLiMat-2-6 & \checkmark \\
R3: Substitution Matrix Validation & TC-SubMat-3-1 to TC-SubMat-3-5 & \checkmark \\
R4/R5: Alignment Correctness & test\_known\_alignment & \checkmark \\
\bottomrule
\end{tabularx}
\end{table}

Key findings:
\begin{itemize}
\item 100\% of functional requirements verified
\item All boundary conditions handled correctly
\item Invalid inputs properly rejected with descriptive errors
\end{itemize}

\section{Nonfunctional Requirements Evaluation}

\subsection{Usability}
\begin{itemize}
\item Successfully passed user survey with domain experts
\item Average rating of 4.2/5 for interface clarity
\item Installation instructions proven effective across platforms
\end{itemize}

\subsection{Performance}
\begin{table}[h]
\centering
\caption{Performance Metrics}
\begin{tabular}{lll}
\toprule
\textbf{Sequence Length} & \textbf{Time (s)} & \textbf{Memory (KB)} \\
\midrule
100bp & 0.02 & 39 \\
1000bp & 0.85 & 390 \\
5000bp & 18.3 & 1950 \\
\bottomrule
\end{tabular}
\end{table}

\subsection{Portability}
\begin{itemize}
\item Verified on:
  \begin{itemize}
  \item Windows 10/11
  \item Linux (Ubuntu 22.04)
  \item macOS Sierra
  \end{itemize}
\item Python 3.8-3.11 compatibility confirmed
\end{itemize}

\section{Unit Testing}

\begin{itemize}
\item 13 unit tests covering all critical modules
\item 100\% of core algorithm paths tested
\item Key test categories:
  \begin{itemize}
  \item Input validation (5 tests)
  \item Matrix operations (4 tests)
  \item Alignment logic (4 tests)
  \end{itemize}
\end{itemize}

\section{Changes Due to Testing}

\begin{itemize}
\item Added maximum sequence length validation (5000bp)
\item Removed matrix symmetry checking after missleading terminology (original documentation used symmetry to describe square matrices)
\item Improved error messages for invalid inputs
\item Optimized file handling based on test failures
\end{itemize}

\section{Automated Testing}

\begin{itemize}
\item Implemented using pytest framework
\item GitHub Actions CI pipeline:
  \begin{itemize}
  \item Runs on all pushes/pull requests
  % \item Tests on Windows/Linux/macOS
  % \item Code coverage reporting
  \end{itemize}
\item Key test statistics:
  \begin{itemize}
  \item Test execution time: ~5.0s
  \item 100\% success rate
  \end{itemize}
\end{itemize}

\newpage

\section{Trace to Requirements}

\autoref{tab:tc-traceability-testcase-and-requirements} contains the mapping of requirements to test cases.

\begin{table}[h!]
\begin{center}
\resizebox{1\textwidth}{!}{ 
\begin{tabular}{ l|c|c|c|c|c|c|c|c|c }
\hline
 & R1 & R2 & R3 & R4 & R5 & NFR2 & NFR4 & NFR5 \\
\hline
test\_main\_empty\_submat & X &  &  &  &  &  &  &  \\
\hline
test\_main\_one\_bp\_seq & X &  &  & X & X &  &  &  \\
\hline
test\_main\_unitary\_submat &  & X & X & X & X &  &  &  \\
\hline
test\_main\_empty\_seq & X &  &  &  &  &  &  &  \\
\hline
test\_valid\_sequences & X &  &  &  &  &  &  &  \\
\hline
test\_empty\_sequence\_a & X &  &  &  &  &  &  &  \\
\hline
test\_invalid\_dna\_chars & X &  &  &  &  &  &  &  \\
\hline
test\_max\_length\_sequences &  & X &  &  &  &  &  & X \\
\hline
test\_exceeds\_max\_length &  & X &  &  &  &  &  &  \\
\hline
test\_valid\_submat &  &  & X &  &  &  &  &  \\
\hline
test\_nonsquare\_matrix &  &  & X &  &  &  &  &  \\
\hline
test\_asymmetric\_matrix &  &  & X &  &  &  &  &  \\
\hline
test\_known\_alignment &  &  &  & X & X &  &  &  \\
\hline
test\_score\_consistency &  &  &  & X & X &  &  &  \\
\hline
test\_performance &  &  &  &  &  &  &  & X \\
\hline
test\_portability &  &  &  &  &  &  & X &  \\
\hline
test\_usability &  &  &  &  &  & X &  &  \\
\hline
\end{tabular}
}
\caption{Traceability Between Test Cases and Requirements}
\label{tab:tc-traceability-testcase-and-requirements}
\end{center}
\end{table}

% ####################
\newpage
\section{Trace to Modules}
\autoref{tab:tc-traceability-testcase-and-modules} contains the mapping of test cases to modules.

\begin{table}[h!]
\begin{center}
\resizebox{1\textwidth}{!}{ 
\begin{tabular}{ l|c|c|c|c|c }
\hline
 & Main & Alignment & File Manager & SequenceData & SubMat \\
\hline
test\_main\_empty\_submat & X & X & X & X &  \\
\hline
test\_main\_one\_bp\_seq & X & X & X & X & X \\
\hline
test\_main\_unitary\_submat & X & X & X & X & X \\
\hline
test\_main\_empty\_seq & X &  & X & X &  \\
\hline
test\_valid\_sequences & X &  & X & X &  \\
\hline
test\_empty\_sequence\_a & X &  & X & X &  \\
\hline
test\_invalid\_dna\_chars & X &  & X & X &  \\
\hline
test\_max\_length\_sequences & X & X & X & X & X \\
\hline
test\_exceeds\_max\_length & X &  & X & X &  \\
\hline
test\_valid\_submat & X &  & X &  & X \\
\hline
test\_nonsquare\_matrix & X &  & X &  & X \\
\hline
test\_asymmetric\_matrix & X &  & X &  & X \\
\hline
test\_known\_alignment & X & X & X & X & X \\
\hline
test\_score\_consistency & X & X & X & X & X \\
\hline
test\_performance & X & X & X & X & X \\
\hline
test\_portability & X &  & X &  &  \\
\hline
test\_usability & X &  & X &  &  \\
\hline
\end{tabular}
}
\caption{Traceability Between Test Cases and Modules}
\label{tab:tc-traceability-testcase-and-modules}
\end{center}
\end{table}
% ####################

\section{Code Coverage Metrics}

\begin{itemize}
\item Overall coverage: 92\%
\item Core algorithm coverage: 100\%
\item File I/O coverage: 85\%
\item Exclusions:
  \begin{itemize}
  \item Error handling for rare OS-level file operations
  \item Deprecated compatibility code paths
  \end{itemize}
\end{itemize}

% \bibliographystyle{plainnat}
% \bibliography{../../refs/References}

\newpage
\section*{Appendix --- Reflection}

\begin{enumerate}
\item \textbf{Successes}:
\begin{itemize}
\item Comprehensive test coverage achieved
\item Automated CI pipeline working effectively
\item Clear requirements traceability established
\end{itemize}

\item \textbf{Challenges}:
\begin{itemize}
\item Initial difficulty testing large sequences
\item Platform-specific file path handling
\item Resolved through test isolation and mocking
\end{itemize}

\item \textbf{Client Feedback}:
\begin{itemize}
\item Domain expert validated scoring logic
\item Peers provided usability input
\end{itemize}

\item \textbf{VnV Plan vs Actual}:
\begin{itemize}
\item Added tests for edge cases not initially considered
\item Revisited the scoring system based on automated tests
\item Changes due to discovering edge cases during testing
\end{itemize}
\end{enumerate}

\end{document}