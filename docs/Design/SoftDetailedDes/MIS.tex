\documentclass[12pt, titlepage]{article}

\usepackage{amsmath, mathtools}

\usepackage[round]{natbib}
\usepackage{amsfonts}
\usepackage{amssymb}
\usepackage{graphicx}
\usepackage{colortbl}
\usepackage{xr}
\usepackage{hyperref}
\usepackage{longtable}
\usepackage{xfrac}
\usepackage{tabularx}
\usepackage{float}
\usepackage{siunitx}
\usepackage{booktabs}
\usepackage{multirow}
\usepackage[section]{placeins}
\usepackage{caption}
\usepackage{fullpage}

\hypersetup{
bookmarks=true,     % show bookmarks bar?
colorlinks=true,       % false: boxed links; true: colored links
linkcolor=red,          % color of internal links (change box color with linkbordercolor)
citecolor=blue,      % color of links to bibliography
filecolor=magenta,  % color of file links
urlcolor=cyan          % color of external links
}

\usepackage{array}

\externaldocument{../../SRS/SRS}

%% Comments

\usepackage{color}

\newif\ifcomments\commentstrue %displays comments
%\newif\ifcomments\commentsfalse %so that comments do not display

\ifcomments
\newcommand{\authornote}[3]{\textcolor{#1}{[#3 ---#2]}}
\newcommand{\todo}[1]{\textcolor{red}{[TODO: #1]}}
\else
\newcommand{\authornote}[3]{}
\newcommand{\todo}[1]{}
\fi

\newcommand{\wss}[1]{\authornote{blue}{SS}{#1}} 
\newcommand{\plt}[1]{\authornote{magenta}{TPLT}{#1}} %For explanation of the template
\newcommand{\an}[1]{\authornote{cyan}{Author}{#1}}

%% Common Parts

\newcommand{\progname}{ProgName} % PUT YOUR PROGRAM NAME HERE
\newcommand{\authname}{Team \#, Team Name
\\ Student 1 name
\\ Student 2 name
\\ Student 3 name
\\ Student 4 name} % AUTHOR NAMES                  

\usepackage{hyperref}
    \hypersetup{colorlinks=true, linkcolor=blue, citecolor=blue, filecolor=blue,
                urlcolor=blue, unicode=false}
    \urlstyle{same}
                                


\begin{document}

\title{Module Interface Specification for \progname{}}

\author{\authname}

\date{\today}

\maketitle

\pagenumbering{roman}

\section{Revision History}

\begin{tabularx}{\textwidth}{p{3cm}p{2cm}X}
\toprule {\bf Date} & {\bf Version} & {\bf Notes}\\
\midrule
March 10, 2025 & 1.0 & Document's first version\\
% Date 2 & 1.1 & Notes\\
\bottomrule
\end{tabularx}

~\newpage

\section{Symbols, Abbreviations and Acronyms}
See SRS Documentation at \href{https://github.com/UGarCil/UGarcil_capstone/blob/main/docs/SRS/SRS.pdf}{SRS Documentation}

% \wss{Also add any additional symbols, abbreviations or acronyms}

\newpage

\tableofcontents

\newpage

\pagenumbering{arabic}

\section{Introduction}

The following document details the Module Interface Specifications for
the \progname software. The software is designed to evaluate the effect of 
given substitution matrices in the quality of the alignment of a set of DNA sequences.

Complementary documents include the System Requirement Specifications
and Module Guide.  The full documentation and implementation can be
found at \url{https://github.com/UGarCil/UGarcil_capstone}.

Many components from the present documentation follow the template
 for a MIS for scientific computing software used in \cite{patel2023module}, 
 \cite{bicket2017module}. These documentations were adapted from the \href{https://website-url.com/mis-template}{MIS template}.

 \section{Notation}

% \wss{You should describe your notation.  You can use what is below as
%   a starting point.}


This section is taken from the \href{https://website-url.com/mis-template}{MIS template}.

\begin{quote}
The structure of the MIS for modules comes from \citet{HoffmanAndStrooper1995},
with the addition that template modules have been adapted from
\cite{GhezziEtAl2003}.  The mathematical notation comes from Chapter 3 of
\citet{HoffmanAndStrooper1995}.  For instance, the symbol := is used for a
multiple assignment statement and conditional rules follow the form $(c_1
\Rightarrow r_1 | c_2 \Rightarrow r_2 | ... | c_n \Rightarrow r_n )$.\dots
\end{quote}

The following table summarizes the primitive data types used by the \progname software. 

\begin{center}
\renewcommand{\arraystretch}{1.2}
\noindent 
\begin{tabular}{l l p{7.5cm}} 
\toprule 
\textbf{Data Type} & \textbf{Notation} & \textbf{Description}\\ 
\midrule
character & char & a single symbol or digit\\
integer & $\mathbb{R}$ & a real (i.e. non complex) number defined within (-$\infty$, $\infty$), which common values between -3.0 to 3.0 \\
natural number & $\mathbb{N}$ & a number without a fractional component in [1, $\infty$) \\
\bottomrule
\end{tabular} 
\end{center}

\noindent
The specification of \progname \ uses some derived data types: sequences, strings, and
tuples. Sequences are lists filled with elements of the same data type. Strings
are sequences of characters. Tuples contain a list of values, potentially of
different types. In addition, \progname \ uses functions, which
are defined by the data types of their inputs and outputs. Local functions are
described by giving their type signature followed by their specification.

\section{Module Decomposition}

The following table is taken directly from the Module Guide document for this project.

\begin{table}[h!]
\centering
\begin{tabular}{p{0.3\textwidth} p{0.6\textwidth}}
\toprule
\textbf{Level 1} & \textbf{Level 2}\\
\midrule

{Hardware-Hiding} & ~ \\
\midrule

\multirow{7}{0.3\textwidth}{Behaviour-Hiding} & Alignment (Needleman-Wunsch) Interface Module\\
& Substitution Matrix Module\\
& Result Concatenation Module\\
& Data Output Module\\
\midrule

\multirow{3}{0.3\textwidth}{Software Decision} & Sequence Data Structure Module\\
& F-Matrix Handler Module\\
& Backtracking Algorithm Module\\
\bottomrule

\end{tabular}
\caption{Module Hierarchy}
\label{TblMH}
\end{table}

\newpage
It is important to highlight the categorization for the Alignment Interface Module in the table above. 
While the use of the Needleman-Wunsch algorithm sits at a high level behavior that defines 
the main architecture of the software, it is also a Software Decision, as future releases of the 
software may opt for other alignment algorithms. Its placement in the documentation as Behaviour-Hiding follows
a functional justification by making the Needleman-Wunsch algorithm a high-behavior decision component.
~\newpage

% #############################################
\section{MIS of Alignment (Needleman-Wunsch) Module} \label{mNW}

\subsection{Module}

needleman\_wunsch

\subsection{Uses}

\begin{itemize}
    \item \textbf{Sequence Data Structure Module}: For representing and manipulating sequences.
    \item \textbf{Substitution Matrix Module}: For accessing substitution matrices and gap penalties.
    \item \textbf{Backtracking Module}: For determining the optimal alignment score.
    \item \textbf{F-Matrix Handler}: To traverse the scores calculated across the pairwise comparison between two sequences.
  \end{itemize}
  
\subsection{Syntax}

\subsubsection{Exported Constants}

None

\subsubsection{Exported Access Programs}

\begin{center}
\begin{tabular}{p{4cm} p{4cm} p{3cm} p{2cm}}
\hline
\textbf{Name} & \textbf{In} & \textbf{Out} & \textbf{Exceptions} \\
\hline
needleman\_wunsch & seq0:str, seq1:str, submat: dict & score:float & See Table 2 and 3 in~\href{https://github.com/UGarCil/UGarcil_capstone/blob/main/docs/VnVPlan/VnVPlan.pdf}{VnV plan}\\
\hline
\end{tabular}
\end{center}

\subsection{Semantics}

\subsubsection{State Variables}

None

\subsubsection{Environment Variables}

None

\subsubsection{Assumptions}

\begin{itemize}
    \item The input sequences (seq0 and seq1) are valid strings containing only the characters A, T, G, C or \_.
    \item The substitution matrix (submat) is a valid dictionary containing penalizing costs and a gap penalty.
  \end{itemize}
  
\subsubsection{Access Routine Semantics}

\noindent \textbf{align\_sequences(seq0: str, seq1: str, submat: dict)}:
\begin{itemize}
    \item \textbf{Transition}: Computes the optimal global alignment between seq0 and seq1 using the Needleman-Wunsch algorithm.
    \item \textbf{Output}: Returns a dictionary containing the alignment score and other relevant results.
    \item \textbf{Exception}: Raises \texttt{InvalidSequenceError} (see Tables 2 and 3 of \href{https://github.com/UGarCil/UGarcil_capstone/blob/main/docs/VnVPlan/VnVPlan.pdf}{VnV plan}) if either sequence contains invalid characters or goes beyond the boundaries accepted by the program.
  \end{itemize}
  
  \subsubsection{Local Functions}

\begin{itemize}
  \item \textbf{backtracking(matrix: List[List[float]], seq0: str, seq1: str) $\rightarrow$ float}:
  \begin{itemize}
      \item Backtracks through the alignment matrix to determine the optimal alignment path.
      \item Returns the final alignment score.
  \end{itemize}
  \item \textbf{evaluateTransTransv(s1: str, s2: str, submat: List[List[float]]) $\rightarrow$ float}:
  \begin{itemize}
      \item Evaluates the cost of a transition or transversion between two nucleotides.
  \end{itemize}
  \item \textbf{evaluate(x: int, y: int, dir: str, submat: List[List[float]], matrix\_F: List[List[float]], seq0: str, seq1: str) $\rightarrow$ float}:
  \begin{itemize}
      \item Evaluates the cost of transitioning from one cell to another in the alignment matrix.
  \end{itemize}
\end{itemize}

\newpage
% #############################################

% #############################################
\section{MIS of Substitution Matrix Module} \label{mSM}

\subsection{Module}

substitution\_matrix

\subsection{Uses}

\begin{itemize}
    \item \textbf{Sequence Data Structure Module}: For representing and manipulating sequences.
    \item \textbf{Alignment (Needleman-Wunsch) Module}: For using substitution matrices in sequence alignment.
\end{itemize}

\subsection{Syntax}

\subsubsection{Exported Constants}

\begin{itemize}
    \item \textbf{penalizingCostOf\_baseline}: A 4x4 matrix representing the baseline penalizing costs for nucleotide comparisons.
    \item \textbf{penalizingCostOf\_JC}: A 4x4 matrix representing the Jukes-Cantor penalizing costs.
    \item \textbf{penalizingCostOf\_K80}: A 4x4 matrix representing the Kimura 1980 penalizing costs.
    \item \textbf{penalizingCostOf\_HKY85}: A 4x4 matrix representing the Hasegawa-Kishino-Yano 1985 penalizing costs.
    \item \textbf{penalizingCostOf\_TN93}: A 4x4 matrix representing the Tamura-Nei 1993 penalizing costs.
\end{itemize}

\subsubsection{Exported Access Programs}

\begin{center}
\begin{tabular}{p{4cm} p{4cm} p{3cm} p{2cm}}
\hline
\textbf{Name} & \textbf{In} & \textbf{Out} & \textbf{Exceptions} \\
\hline
get\_substitution\_matrix & name: str & submat: dict & see Table 4 in \href{https://github.com/UGarCil/UGarcil_capstone/blob/main/docs/VnVPlan/VnVPlan.pdf}{VnV Plan} \\
\hline
\end{tabular}
\end{center}

\subsection{Semantics}

\subsubsection{State Variables}

None

\subsubsection{Environment Variables}

None

\subsubsection{Assumptions}

\begin{itemize}
    \item The substitution matrices are valid 4x4 matrices containing penalizing costs for nucleotide comparisons.
    \item The gap penalty is a valid float value.
\end{itemize}

\subsubsection{Access Routine Semantics}

\noindent \textbf{get\_substitution\_matrix(name: str) $\rightarrow$ dict}:
\begin{itemize}
    \item \textbf{Transition}: Retrieves the substitution matrix and gap penalty associated with the given name.
    \item \textbf{Output}: Returns a dictionary containing the substitution matrix and gap penalty.
    \item \textbf{Exception}: Raises \texttt{InvalidMatrixError} (see Table 4 of \href{https://github.com/UGarCil/UGarcil_capstone/blob/main/docs/VnVPlan/VnVPlan.pdf}{VnV Plan} for further details) if the matrix dimensions are invalid.
\end{itemize}

\subsubsection{Local Functions}

None

\newpage
% #############################################

% #############################################
\section{MIS of Result Concatenation Module} \label{mRC}

\subsection{Module}

result\_concatenation

\subsection{Uses}

\begin{itemize}
    \item \textbf{Alignment (Needleman-Wunsch) Module}: For performing sequence alignments and obtaining alignment scores.
    \item \textbf{Substitution Matrix Module}: For accessing substitution matrices and gap penalties.
    \item \textbf{Pandas Library}: For organizing and displaying results in a tabular format.
\end{itemize}

\subsection{Syntax}

\subsubsection{Exported Constants}

None

\subsubsection{Exported Access Programs}

\begin{center}
\begin{tabular}{p{4cm} p{4cm} p{3cm} p{2cm}}
\hline
\textbf{Name} & \textbf{In} & \textbf{Out} & \textbf{Exceptions} \\
\hline
main & seq0: str, seq1: str & result: List[dict] &  \\
\hline
\end{tabular}
\end{center}

\subsection{Semantics}

\subsubsection{State Variables}

None

\subsubsection{Environment Variables}

None

\subsubsection{Assumptions}

\begin{itemize}
    \item As stated in Alignment (Needleman-Wunsch) and Substitution Matrix modules.
    \item The Pandas library is available for data manipulation.
\end{itemize}

\subsubsection{Access Routine Semantics}

\noindent \textbf{main(seq0: str, seq1: str) $\rightarrow$ pandas.DataFrame}:
\begin{itemize}
    \item \textbf{Transition}: Computes the alignment scores for the given sequences using all available substitution matrices.
    \item \textbf{Output}: Returns a list of dictionaries, where each dictionary contains the name of the substitution matrix and the corresponding alignment score.
    \item \textbf{Exception}: Raises \texttt{InvalidSequenceError} if either sequence contains invalid characters.
\end{itemize}

\subsubsection{Local Functions}

None

\newpage
% #############################################

% #############################################
\section{MIS of Data Output Module} \label{mDO}

\subsection{Module}

data\_output

\subsection{Uses}

\begin{itemize}
    \item \textbf{Result Concatenation Module}: For providing the alignment results to be displayed.
    \item \textbf{Pandas Library}: For organizing and displaying results in a tabular format.
\end{itemize}

\subsection{Syntax}

\subsubsection{Exported Constants}

None

\subsubsection{Exported Access Programs}

\begin{center}
\begin{tabular}{p{4cm} p{4cm} p{3cm} p{2cm}}
\hline
\textbf{Name} & \textbf{In} & \textbf{Out} & \textbf{Exceptions} \\
\hline
display\_results & result: List[dict] & None & InvalidResultError \\
\hline
\end{tabular}
\end{center}

\subsection{Semantics}

\subsubsection{State Variables}

None

\subsubsection{Environment Variables}

None

\subsubsection{Assumptions}

\begin{itemize}
    \item The input result is a valid list of dictionaries, where each dictionary contains the name of the substitution matrix and the corresponding alignment score.
    \item The Pandas library is available for data manipulation and display.
\end{itemize}

\subsubsection{Access Routine Semantics}

\noindent \textbf{display\_results(result: List[dict]) $\rightarrow$ None}:
\begin{itemize}
    \item \textbf{Transition}: Converts the list of alignment results into a Pandas DataFrame and displays it in a tabular format.
    \item \textbf{Output}: None (the results are displayed directly to the console or output medium).
    \item \textbf{Exception}: Raises \texttt{InvalidResultError} if the input result is not a valid list of dictionaries.
\end{itemize}

\subsubsection{Local Functions}

None

\newpage
% #############################################
% #############################################
\section{MIS of Sequence Data Structure Module} \label{mSDS}

\subsection{Module}

sequence\_data\_structure

\subsection{Uses}

None

\subsection{Syntax}

\subsubsection{Exported Constants}

None

\subsubsection{Exported Access Programs}

\begin{center}
\begin{tabular}{p{4cm} p{4cm} p{3cm} p{2cm}}
\hline
\textbf{Name} & \textbf{In} & \textbf{Out} & \textbf{Exceptions} \\
\hline
validate\_sequence & seq: str & bool & InvalidSequenceError \\
\hline
\end{tabular}
\end{center}

\subsection{Semantics}

\subsubsection{State Variables}

None

\subsubsection{Environment Variables}

None

\subsubsection{Assumptions}

\begin{itemize}
    \item Sequences are represented as strings containing only the characters A, T, G, C, or \_.
    \item Sequences are non-empty and valid for alignment purposes.
\end{itemize}

\subsubsection{Access Routine Semantics}

\noindent \textbf{validate\_sequence(seq: str) $\rightarrow$ bool}:
\begin{itemize}
    \item \textbf{Transition}: Validates the input sequence to ensure it contains only valid characters (A, T, G, C, or \_).
    \item \textbf{Output}: Returns \texttt{True} if the sequence is valid, otherwise raises \texttt{InvalidSequenceError}.
    \item \textbf{Exception}: Raises \texttt{InvalidSequenceError} (see Table 2 and 3 from \href{https://github.com/UGarCil/UGarcil_capstone/blob/main/docs/VnVPlan/VnVPlan.pdf}{VnV documentation}) if the sequence contains invalid characters.
\end{itemize}

\subsubsection{Local Functions}

None

\newpage
% #############################################
% #############################################
\section{MIS of F-Matrix Handler Module} \label{mFIM}

\subsection{Module}

matrix

\subsection{Uses}

\begin{itemize}
    \item \textbf{Sequence Data Structure Module}: For accessing sequences to be aligned.
    \item \textbf{Substitution Matrix Module}: For accessing substitution matrices and gap penalties.
\end{itemize}

\subsection{Syntax}

\subsubsection{Exported Constants}

None

\subsubsection{Exported Access Programs}

\begin{center}
\begin{tabular}{p{4cm} p{4cm} p{3cm} p{2cm}}
\hline
\textbf{Name} & \textbf{In} & \textbf{Out} & \textbf{Exceptions} \\
\hline
initialize\_matrix & seq0: str, seq1: str & matrix: List[List[float]] & InvalidSequenceError \\
fill\_matrix & seq0: str, seq1: str, submat: List[List[float]] & matrix: List[List[float]] & InvalidSequenceError \\
\hline
\end{tabular}
\end{center}

\subsection{Semantics}

\subsubsection{State Variables}

None

\subsubsection{Environment Variables}

None

\subsubsection{Assumptions}

\begin{itemize}
    \item The input sequences (seq0 and seq1) are valid strings containing only the characters A, T, G, C, or \_.
    \item The substitution matrix (submat) is a valid 4x4 matrix of penalizing costs.
    \item The gap penalty is a valid float value.
\end{itemize}

\subsubsection{Access Routine Semantics}

\noindent \textbf{initialize\_matrix(seq0: str, seq1: str) $\rightarrow$ List[List[float]]}:
\begin{itemize}
    \item \textbf{Transition}: Initializes the F-matrix with dimensions based on the lengths of seq0 and seq1.
    \item \textbf{Output}: Returns a 2D list (matrix) initialized with zeros.
    \item \textbf{Exception}: Raises \texttt{InvalidSequenceError} if either sequence contains invalid characters.
\end{itemize}

\noindent \textbf{fill\_matrix(seq0: str, seq1: str, submat: List[List[float]]) $\rightarrow$ List[List[float]]}:
\begin{itemize}
    \item \textbf{Transition}: Fills the F-matrix with alignment scores based on the sequences and substitution matrix.
    \item \textbf{Output}: Returns the filled F-matrix.
    \item \textbf{Exception}: Raises \texttt{InvalidSequenceError} if either sequence contains invalid characters.
\end{itemize}

\subsubsection{Local Functions}

None

\newpage
% #############################################
% #############################################
\section{MIS of Backtracking Algorithm Module} \label{mBA}

\subsection{Module}

backtracking

\subsection{Uses}

\begin{itemize}
    \item \textbf{F-Matrix Initialization and Manipulation Module}: For accessing the filled F-matrix.
    \item \textbf{Sequence Data Structure Module}: For accessing the sequences to be aligned.
\end{itemize}

\subsection{Syntax}

\subsubsection{Exported Constants}

None

\subsubsection{Exported Access Programs}

\begin{center}
\begin{tabular}{p{4cm} p{4cm} p{3cm} p{2cm}}
\hline
\textbf{Name} & \textbf{In} & \textbf{Out} & \textbf{Exceptions} \\
\hline
backtracking & matrix: List[List[float]], seq0: str, seq1: str & float & InvalidMatrixError \\
\hline
\end{tabular}
\end{center}

\subsection{Semantics}

\subsubsection{State Variables}

None

\subsubsection{Environment Variables}

None

\subsubsection{Assumptions}

\begin{itemize}
    \item The input matrix is a valid 2D list (F-matrix) filled with alignment scores.
    \item The input sequences (seq0 and seq1) are valid strings containing only the characters A, T, G, C, or \_.
\end{itemize}

\subsubsection{Access Routine Semantics}

\noindent \textbf{backtracking(matrix: List[List[float]], seq0: str, seq1: str) $\rightarrow$ float}:
\begin{itemize}
    \item \textbf{Transition}: Backtracks through the F-matrix to determine the optimal alignment path (i.e. the highest value out of the global alignment).
    \item \textbf{Output}: Returns the final alignment score.
    \item \textbf{Exception}: Raises \texttt{InvalidMatrixError} if the input matrix is not a valid 2D list or if the sequences are invalid.
\end{itemize}

\subsubsection{Local Functions}

None

\newpage
% #############################################


\bibliographystyle {plainnat}
\bibliography {../../../refs/References}

% \newpage

% \section{Appendix} \label{Appendix}

% \wss{Extra information if required}

% \newpage{}

% \section*{Appendix --- Reflection}

% \wss{Not required for CAS 741 projects}

% The information in this section will be used to evaluate the team members on the
% graduate attribute of Problem Analysis and Design.

% The purpose of reflection questions is to give you a chance to assess your own
learning and that of your group as a whole, and to find ways to improve in the
future. Reflection is an important part of the learning process.  Reflection is
also an essential component of a successful software development process.  

Reflections are most interesting and useful when they're honest, even if the
stories they tell are imperfect. You will be marked based on your depth of
thought and analysis, and not based on the content of the reflections
themselves. Thus, for full marks we encourage you to answer openly and honestly
and to avoid simply writing ``what you think the evaluator wants to hear.''

Please answer the following questions.  Some questions can be answered on the
team level, but where appropriate, each team member should write their own
response:


% \begin{enumerate}
%   \item What went well while writing this deliverable? 
%   \item What pain points did you experience during this deliverable, and how
%     did you resolve them?
%   \item Which of your design decisions stemmed from speaking to your client(s)
%   or a proxy (e.g. your peers, stakeholders, potential users)? For those that
%   were not, why, and where did they come from?
%   \item While creating the design doc, what parts of your other documents (e.g.
%   requirements, hazard analysis, etc), it any, needed to be changed, and why?
%   \item What are the limitations of your solution?  Put another way, given
%   unlimited resources, what could you do to make the project better? (LO\_ProbSolutions)
%   \item Give a brief overview of other design solutions you considered.  What
%   are the benefits and tradeoffs of those other designs compared with the chosen
%   design?  From all the potential options, why did you select the documented design?
%   (LO\_Explores)
% \end{enumerate}


\end{document}