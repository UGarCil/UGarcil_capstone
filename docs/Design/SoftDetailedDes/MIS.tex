\documentclass[12pt, titlepage]{article}

\usepackage{amsmath, mathtools}

\usepackage[round]{natbib}
\usepackage{amsfonts}
\usepackage{amssymb}
\usepackage{graphicx}
\usepackage{colortbl}
\usepackage{xr}
\usepackage{hyperref}
\usepackage{longtable}
\usepackage{xfrac}
\usepackage{tabularx}
\usepackage{float}
\usepackage{siunitx}
\usepackage{booktabs}
\usepackage{multirow}
\usepackage[section]{placeins}
\usepackage{caption}
\usepackage{fullpage}

\hypersetup{
bookmarks=true,     % show bookmarks bar?
colorlinks=true,       % false: boxed links; true: colored links
linkcolor=red,          % color of internal links (change box color with linkbordercolor)
citecolor=blue,      % color of links to bibliography
filecolor=magenta,  % color of file links
urlcolor=cyan          % color of external links
}

\usepackage{array}

\externaldocument{../../SRS/SRS}

%% Comments

\usepackage{color}

\newif\ifcomments\commentstrue %displays comments
%\newif\ifcomments\commentsfalse %so that comments do not display

\ifcomments
\newcommand{\authornote}[3]{\textcolor{#1}{[#3 ---#2]}}
\newcommand{\todo}[1]{\textcolor{red}{[TODO: #1]}}
\else
\newcommand{\authornote}[3]{}
\newcommand{\todo}[1]{}
\fi

\newcommand{\wss}[1]{\authornote{blue}{SS}{#1}} 
\newcommand{\plt}[1]{\authornote{magenta}{TPLT}{#1}} %For explanation of the template
\newcommand{\an}[1]{\authornote{cyan}{Author}{#1}}

%% Common Parts

\newcommand{\progname}{ProgName} % PUT YOUR PROGRAM NAME HERE
\newcommand{\authname}{Team \#, Team Name
\\ Student 1 name
\\ Student 2 name
\\ Student 3 name
\\ Student 4 name} % AUTHOR NAMES                  

\usepackage{hyperref}
    \hypersetup{colorlinks=true, linkcolor=blue, citecolor=blue, filecolor=blue,
                urlcolor=blue, unicode=false}
    \urlstyle{same}
                                


\begin{document}

\title{Module Interface Specification for \progname{}}

\author{\authname}

\date{\today}

\maketitle

\pagenumbering{roman}

\section{Revision History}

\begin{tabularx}{\textwidth}{p{3cm}p{2cm}X}
\toprule {\bf Date} & {\bf Version} & {\bf Notes}\\
\midrule
March 10, 2025 & 1.0 & Document's first version\\
April 4th, 2025 & 1.1 & Document's updates after changes on modular design\\
% Date 2 & 1.1 & Notes\\
\bottomrule
\end{tabularx}

~\newpage

\section{Symbols, Abbreviations and Acronyms}
See SRS Documentation at \href{https://github.com/UGarCil/UGarcil_capstone/blob/main/docs/SRS/SRS.pdf}{SRS Documentation}

% \wss{Also add any additional symbols, abbreviations or acronyms}

\newpage

\tableofcontents

\newpage

\pagenumbering{arabic}

\section{Introduction}

The following document details the Module Interface Specifications for
the \progname software. The software is designed to evaluate the effect of 
given substitution matrices in the quality of the alignment of a set of DNA sequences.

Complementary documents include the System Requirement Specifications
and Module Guide.  The full documentation and implementation can be
found at \url{https://github.com/UGarCil/UGarcil_capstone}.

Many components from the present documentation follow the template
 for a MIS for scientific computing software used in \cite{patel2023module}, 
 \cite{bicket2017module}. These documentations were adapted from the \href{https://website-url.com/mis-template}{MIS template}.

 \section{Notation}

% \wss{You should describe your notation.  You can use what is below as
%   a starting point.}


This section is taken from the \href{https://website-url.com/mis-template}{MIS template}.

\begin{quote}
The structure of the MIS for modules comes from \citet{HoffmanAndStrooper1995},
with the addition that template modules have been adapted from
\cite{GhezziEtAl2003}.  The mathematical notation comes from Chapter 3 of
\citet{HoffmanAndStrooper1995}.  For instance, the symbol := is used for a
multiple assignment statement and conditional rules follow the form $(c_1
\Rightarrow r_1 | c_2 \Rightarrow r_2 | ... | c_n \Rightarrow r_n )$.\dots
\end{quote}

The following table summarizes the primitive data types used by the \progname software. 

\begin{center}
\renewcommand{\arraystretch}{1.2}
\noindent 
\begin{tabular}{l l p{7.5cm}} 
\toprule 
\textbf{Data Type} & \textbf{Notation} & \textbf{Description}\\ 
\midrule
character & char & a single symbol or digit\\
integer & $\mathbb{R}$ & a real (i.e. non complex) number defined within (-$\infty$, $\infty$), which common values between -3.0 to 3.0 \\
natural number & $\mathbb{N}$ & a number without a fractional component in [1, $\infty$) \\
\bottomrule
\end{tabular} 
\end{center}

\noindent
The specification of \progname \ uses some derived data types: sequences, strings, and
tuples. Sequences are lists filled with elements of the same data type. Strings
are sequences of characters. Tuples contain a list of values, potentially of
different types. In addition, \progname \ uses functions, which
are defined by the data types of their inputs and outputs. Local functions are
described by giving their type signature followed by their specification.

\section{Module Decomposition}

The following table is taken directly from the Module Guide document for this project.

\begin{table}[h!]
    \centering
    \begin{tabular}{p{0.3\textwidth} p{0.6\textwidth}}
    \toprule
    \textbf{Level 1} & \textbf{Level 2}\\
    \midrule
    
    Hardware-Hiding & ~ \\
    \midrule
    
    Behaviour-Hiding & Alignment (Needleman-Wunsch) Interface Module\\
    & Control Manager Module\\
    \midrule
    
    Software Decision & Substitution Matrix Module\\
    & File Manager Module\\
    & Sequence Data Structure Module\\
    \bottomrule
    
    \end{tabular}
    \caption{Module Hierarchy}
    \label{TblMH}
\end{table}


\newpage
It is important to highlight the categorization for the Alignment Interface Module in the table above. 
While the use of the Needleman-Wunsch algorithm sits at a high level behavior that defines 
the main architecture of the software, it is also a Software Decision, as future releases of the 
software may opt for other alignment algorithms. Its placement in the documentation as Behaviour-Hiding follows
a functional justification by making the Needleman-Wunsch algorithm a high-behavior decision component.
~\newpage

% #############################################
\section{MIS of Alignment (Needleman-Wunsch) Module} \label{mNW}

\subsection{Module}

NeedlemanWunsch

\subsection{Uses}

\begin{itemize}
    \item \textbf{Sequence Data Structure Module (mSDS)}: For validated biological sequence input
    \item \textbf{Substitution Matrix Module (mSM)}: For accessing substitution matrices and gap penalties
\end{itemize}

\subsection{Syntax}

\subsubsection{Exported Constants}

None

\subsubsection{Exported Access Programs}

\begin{center}
\begin{tabular}{p{4cm} p{4cm} p{3cm} p{2cm}}
\hline
\textbf{Name} & \textbf{In} & \textbf{Out} & \textbf{Exceptions} \\
\hline
\texttt{\_\_init\_\_} & seq\_a: str, seq\_b: str, gap\_penalty: float = -2 & NeedlemanWunsch instance & - \\
\hline
\texttt{\_\_call\_\_} & submat: dict & float (alignment score) & AlignmentError \\
\hline
\texttt{align} & submat: dict & float (alignment score) & AlignmentError \\
\hline
\texttt{get\_aligned\_sequences} & - & tuple (str, str) & - \\
\hline
\end{tabular}
\end{center}

\subsection{Semantics}

\subsubsection{State Variables}

\begin{itemize}
    \item \texttt{seq\_a}: First input sequence
    \item \texttt{seq\_b}: Second input sequence
    \item \texttt{gap\_penalty}: Gap insertion penalty score
    \item \texttt{submat}: Active substitution matrix
    \item \texttt{seq\_a\_aligned}: Aligned version of seq\_a
    \item \texttt{seq\_b\_aligned}: Aligned version of seq\_b
\end{itemize}

\subsubsection{Environment Variables}

None

\subsubsection{Assumptions}

\begin{itemize}
    \item Input sequences contain only characters \{A, T, G, C, \_\}
    \item Sequences have been pre-validated by SequenceData module
    \item Substitution matrix is a valid 4x4 matrix for nucleotides
    \item Gap penalty is a negative float value
\end{itemize}

\subsubsection{Access Routine Semantics}

\noindent \textbf{NeedlemanWunsch.\_\_init\_\_(seq\_a, seq\_b, gap\_penalty=-2)}:
\begin{itemize}
    \item Initializes alignment processor with sequences and gap penalty
    \item Transition: Sets instance variables
\end{itemize}

\noindent \textbf{NeedlemanWunsch.\_\_call\_\_(submat)}:
\begin{itemize}
    \item Output: Alignment score (float)
    \item Exception: AlignmentError if sequences cannot be aligned
    \item Delegates to align() method
\end{itemize}

\noindent \textbf{NeedlemanWunsch.align(submat)}:
\begin{itemize}
    \item Transition: 
    \begin{enumerate}
        \item Sets active substitution matrix
        \item Initializes dynamic programming matrix
        \item Computes alignment scores
        \item Performs traceback to determine optimal alignment
    \end{enumerate}
    \item Output: Final alignment score (float)
    \item Exception: AlignmentError if alignment fails
\end{itemize}

\noindent \textbf{NeedlemanWunsch.get\_aligned\_sequences()}:
\begin{itemize}
    \item Output: Tuple of aligned sequences (str, str)
    \item Requires: align() must have been called first
\end{itemize}

\subsubsection{Local Functions}

\begin{itemize}
    \item \textbf{evaluate\_substitution(s1: str, s2: str) $\rightarrow$ float}:
    \begin{itemize}
        \item Looks up substitution score in matrix
        \item Handles both matrix formats (direct 4x4 or parameterized)
        \item Returns score for nucleotide pair
    \end{itemize}
    
    \item \textbf{compute\_alignment\_matrix(matrix) $\rightarrow$ List[List[float]]}:
    \begin{itemize}
        \item Fills dynamic programming matrix
        \item Applies recurrence relation:
        \begin{equation}
            F(i,j) = \max\begin{cases}
                F(i-1,j-1) + s(x_i,y_j) \\
                F(i-1,j) + d \\
                F(i,j-1) + d
            \end{cases}
        \end{equation}
    \end{itemize}
    
    \item \textbf{backtrack(matrix) $\rightarrow$ float}:
    \begin{itemize}
        \item Traces optimal path through DP matrix
        \item Sets seq\_a\_aligned and seq\_b\_aligned
        \item Returns final alignment score
    \end{itemize}
\end{itemize}

\subsection{Considerations}

\begin{itemize}
    \item Time Complexity: O(mn) for sequences of length m and n
    \item Space Complexity: O(mn) for full matrix storage
    \item Handles both traditional and parameterized substitution matrices
    \item Inserts gap penalties through constant gap penalty
\end{itemize}

\newpage
% #############################################

\section{MIS of Control Manager Module} \label{mCM}

\subsection{Module}

main

\subsection{Uses}

\begin{itemize}
    \item \textbf{File Manager Module}: For file I/O operations
    \item \textbf{Alignment Module}: For sequence alignment
    \item \textbf{Sequence Data Structure Module}: For sequence handling
    \item \textbf{Substitution Matrix Module}: For matrix operations
\end{itemize}

\subsection{Syntax}

\subsubsection{Exported Constants}

None

\subsubsection{Exported Access Programs}

\begin{center}
\begin{tabular}{p{4cm} p{4cm} p{3cm} p{2cm}}
\hline
\textbf{Name} & \textbf{In} & \textbf{Out} & \textbf{Exceptions} \\
\hline
main & seq\_data: SequenceData, matrix\_bench: SubMat & list[dict] & AlignmentError \\
\hline
\end{tabular}
\end{center}

\subsection{Semantics}

\subsubsection{State Variables}

None

\subsubsection{Environment Variables}

None

\subsubsection{Assumptions}

\begin{itemize}
    \item Input files exist and are properly formatted
    \item Required modules are properly initialized
    \item Output directory is writable
\end{itemize}

\subsubsection{Access Routine Semantics}

\noindent \textbf{main(seq\_data, matrix\_bench)}:
\begin{itemize}
    \item Transition:
    \begin{enumerate}
        \item Initializes NeedlemanWunsch alignment manager
        \item Iterates through each substitution matrix
        \item Computes alignment score for each matrix
        \item Collects results
    \end{enumerate}
\end{itemize}

\subsection{Workflow}

The control flow follows this sequence:

\begin{enumerate}
    \item Sequence data is loaded via SequenceData module
    \item Substitution matrices are loaded via SubMat module
    \item For each substitution matrix:
    \begin{enumerate}
        \item Alignment is performed via NeedlemanWunsch
        \item Results are collected
    \end{enumerate}
    \item Results are exported via FileManager
\end{enumerate}

% \begin{figure}[H]
% \centering
% \includegraphics[width=0.8\textwidth]{workflow.png}
% \caption{Control flow diagram}
% \label{fig:workflow}
% \end{figure}

\subsection{Considerations}

\begin{itemize}
    \item Acts as the system's main coordinator
    \item Stateless - maintains no internal state between runs
    \item Delegates all specialized operations to other modules
    \item Handles the sequencing of operations but not their implementation
\end{itemize}

% #############################################
\section{MIS of Substitution Matrix Module} \label{mSM}

\subsection{Module}

substitution\_matrix

\subsection{Uses}

\begin{itemize}
    \item \textbf{File Manager Module}: For loading matrix data from files
\end{itemize}

\subsection{Syntax}

\subsubsection{Exported Constants}

\begin{itemize}
    \item \textbf{penalizingCostOf\_}: A 4x4 matrix representing the baseline penalizing costs for nucleotide comparisons.
\end{itemize}

\subsubsection{Exported Access Programs}

\begin{center}
\begin{tabular}{p{4cm} p{4cm} p{3cm} p{2cm}}
\hline
\textbf{Name} & \textbf{In} & \textbf{Out} & \textbf{Exceptions} \\
\hline
get\_substitution\_matrix & name: str & submat: dict & InvalidMatrixError \\
validate\_benchmark & benchmark: list[dict] & - & InvalidMatrixError \\
get\_matrix\_names & - & list[str] & - \\
\hline
\end{tabular}
\end{center}

\subsection{Semantics}

\subsubsection{State Variables}

\begin{itemize}
    \item \texttt{data}: List of matrix dictionaries containing:
    \begin{itemize}
        \item \texttt{NAME}: String identifier
        \item \texttt{PENALIZING\_COSTS}: 4x4 numerical matrix
    \end{itemize}
\end{itemize}

\subsubsection{Environment Variables}

None

\subsubsection{Assumptions}

\begin{itemize}
    \item Input files follow the format:
    \begin{verbatim}
    >MatrixName
    A,T,G,C
    T,A,G,C
    G,T,A,C
    C,G,T,A
    \end{verbatim}
    \item All matrices are 4x4 and square
    \item Nucleotide order is fixed as [A, T, G, C] for index mapping
\end{itemize}

\subsubsection{Access Routine Semantics}

\noindent \textbf{get\_substitution\_matrix(name: str) $\rightarrow$ dict}:
\begin{itemize}
    \item Retrieves matrix dictionary by name from \texttt{data}
    \item \textbf{Output}: Dictionary with keys:
    \begin{itemize}
        \item \texttt{NAME}: Matrix identifier
        \item \texttt{PENALIZING\_COSTS}: 4x4 scoring matrix
    \end{itemize}
\end{itemize}

\noindent \textbf{validate\_benchmark(benchmark: list[dict])}:
\begin{itemize}
    \item \textbf{Exception}: Raises \texttt{InvalidMatrixError} if:
    \begin{itemize}
        \item Any matrix is not 4x4
        \item Contains non-numeric values
    \end{itemize}
    Returns boolean indicating validity
\end{itemize}

\noindent \textbf{get\_matrix\_names() $\rightarrow$ list[str]}:
\begin{itemize}
    \item \textbf{Output}: List of all loaded matrix names
\end{itemize}

\subsubsection{Local Functions}

None

\newpage
% #############################################

% #############################################
\section{MIS of File Manager Module} \label{mSFM}

\subsection{Module}

file\_manager

\subsection{Uses}

\begin{itemize}
    \item \textbf{Software Hardware Module}: For interaction with I/O services
    \item \textbf{Control Manager Module}: For file management operations and file path information
\end{itemize}

\subsection{Syntax}

\subsubsection{Exported Constants}

\begin{itemize}
    \item None
\end{itemize}

\subsubsection{Exported Access Programs}

\begin{center}
\begin{tabular}{p{4cm} p{4cm} p{3cm} p{2cm}}
\hline
\textbf{Name} & \textbf{In} & \textbf{Out} & \textbf{Exceptions} \\
\hline
read\_sequence\_file & file\_path: str & tuple (str, str) & ValueError \\
read\_submat\_file & file\_path: str & list[dict] & ValueError \\
export & data: list[dict], file\_path: str & - & IOError \\
\hline
\end{tabular}
\end{center}

\subsection{Semantics}

\subsubsection{State Variables}

None

\subsubsection{Environment Variables}

None

\subsubsection{Assumptions}

\begin{itemize}
    \item Input FASTA files follow format:
    \begin{verbatim}
    >Sequence1
    ATGC
    >Sequence2
    TGCA
    \end{verbatim}
    \item Substitution matrix files follow format:
    \begin{verbatim}
    >MatrixName
    1.0,-0.33,-0.33,-0.33
    -0.33,1.0,-0.33,-0.33
    -0.33,-0.33,1.0,-0.33
    -0.33,-0.33,-0.33,1.0
    \end{verbatim}
    \item Output directories are writable
\end{itemize}

\subsubsection{Access Routine Semantics}

\noindent \textbf{read\_sequence\_file(file\_path: str) $\rightarrow$ tuple}:
\begin{itemize}
    \item Reads and validates FASTA file
    \item \textbf{Output}: Tuple of two nucleotide sequences (str, str)
    \item \textbf{Exception}: Raises \texttt{ValueError} if:
    \begin{itemize}
        \item Missing sequence headers
        \item Invalid number of sequences
        \item Zero-length sequences
    \end{itemize}
\end{itemize}

\noindent \textbf{read\_submat\_file(file\_path: str) $\rightarrow$ list[dict]}:
\begin{itemize}
    \item Loads and validates substitution matrices
    \item \textbf{Output}: List of matrix dictionaries with:
    \begin{itemize}
        \item \texttt{NAME}: String identifier
        \item \texttt{PENALIZING\_COSTS}: 4x4 numerical matrix
    \end{itemize}
    \item \textbf{Exception}: Raises \texttt{ValueError} if:
    \begin{itemize}
        \item Non-square matrices
        \item Non-numeric values
        \item Incomplete matrix data
    \end{itemize}
\end{itemize}

\noindent \textbf{export(data: list[dict], file\_path: str)}:
\begin{itemize}
    \item Writes benchmark results to CSV
    \item \textbf{Output}: None (creates file at specified path)
    \item \textbf{Exception}: Raises \texttt{IOError} if file cannot be written
\end{itemize}

\subsubsection{Local Functions}

None

\subsection{Considerations}

\begin{itemize}
    \item UTF-8 encoding enforced for all text files
    \item CSV output includes headers: "matrix" and "score"
    \item Paths are resolved relative to user's designated directory
\end{itemize}

\newpage

% #############################################

% #############################################
\section{MIS of Sequence Data Structure Module} \label{mSDS}

\subsection{Module}

sequence\_data\_structure

\subsection{Uses}

\begin{itemize}
    \item \textbf{File Manager Module}: For loading sequence data from files
\end{itemize}

\subsection{Syntax}

\subsubsection{Exported Constants}

None

\subsubsection{Exported Access Programs}

\begin{center}
\begin{tabular}{p{4cm} p{4cm} p{3cm} p{2cm}}
\hline
\textbf{Name} & \textbf{In} & \textbf{Out} & \textbf{Exceptions} \\
\hline
\texttt{\_\_init\_\_} & read\_file\_path: str & SequenceData instance & ValueError \\
\texttt{validate\_sequence} & seq: str & - & ValueError \\
\texttt{get\_sequences} & - & tuple (str, str) & - \\
\hline
\end{tabular}
\end{center}

\subsection{Semantics}

\subsubsection{State Variables}

\begin{itemize}
    \item \texttt{seq\_a}: First nucleotide sequence (5' to 3')
    \item \texttt{seq\_b}: Second nucleotide sequence (5' to 3')
\end{itemize}

\subsubsection{Environment Variables}

None

\subsubsection{Assumptions}

\begin{itemize}
    \item Sequences contain only characters \{A, T, G, C, \_\}
    \item Sequences are non-empty (length $\geq$ 1)
    \item Input files contain exactly two sequences
    \item Underscore (\_) represents gap characters
\end{itemize}

\subsubsection{Access Routine Semantics}

\noindent \textbf{SequenceData.\_\_init\_\_(read\_file\_path: str)}:
\begin{itemize}
    \item Loads sequences via \texttt{read\_sequence\_file()}
    \item Validates sequences using \texttt{validate\_sequence()}
    \item Initializes \texttt{seq\_a} and \texttt{seq\_b}
    \item \textbf{Exception}: Raises \texttt{ValueError} for:
    \begin{itemize}
        \item Invalid FASTA format
        \item Invalid nucleotide characters
        \item Unequal sequence counts
    \end{itemize}
\end{itemize}

\noindent \textbf{validate\_sequence(seq: str)}:
\begin{itemize}
    \item Checks sequence validity
    \item \textbf{Exception}: Raises \texttt{ValueError} if:
    \begin{itemize}
        \item Contains non-\{A,T,G,C,\_\} characters
        \item Zero-length sequence
    \end{itemize}
\end{itemize}

\noindent \textbf{get\_sequences() $\rightarrow$ tuple}:
\begin{itemize}
    \item \textbf{Output}: Returns (\texttt{seq\_a}, \texttt{seq\_b}) pair
\end{itemize}

\subsubsection{Local Functions}

None

\subsection{Considerations}

\begin{itemize}
    \item Case-sensitive for nucleotide characters
    \item No support for ambiguous bases (N, R, Y, etc.)
    \item Gap character (\_) must be explicitly included
    \item Validation occurs during initialization
\end{itemize}

\newpage
% #############################################


\bibliographystyle {plainnat}
\bibliography {../../../refs/References}

% \newpage

% \section{Appendix} \label{Appendix}

% \wss{Extra information if required}

% \newpage{}

% \section*{Appendix --- Reflection}

% \wss{Not required for CAS 741 projects}

% The information in this section will be used to evaluate the team members on the
% graduate attribute of Problem Analysis and Design.

% The purpose of reflection questions is to give you a chance to assess your own
learning and that of your group as a whole, and to find ways to improve in the
future. Reflection is an important part of the learning process.  Reflection is
also an essential component of a successful software development process.  

Reflections are most interesting and useful when they're honest, even if the
stories they tell are imperfect. You will be marked based on your depth of
thought and analysis, and not based on the content of the reflections
themselves. Thus, for full marks we encourage you to answer openly and honestly
and to avoid simply writing ``what you think the evaluator wants to hear.''

Please answer the following questions.  Some questions can be answered on the
team level, but where appropriate, each team member should write their own
response:


% \begin{enumerate}
%   \item What went well while writing this deliverable? 
%   \item What pain points did you experience during this deliverable, and how
%     did you resolve them?
%   \item Which of your design decisions stemmed from speaking to your client(s)
%   or a proxy (e.g. your peers, stakeholders, potential users)? For those that
%   were not, why, and where did they come from?
%   \item While creating the design doc, what parts of your other documents (e.g.
%   requirements, hazard analysis, etc), it any, needed to be changed, and why?
%   \item What are the limitations of your solution?  Put another way, given
%   unlimited resources, what could you do to make the project better? (LO\_ProbSolutions)
%   \item Give a brief overview of other design solutions you considered.  What
%   are the benefits and tradeoffs of those other designs compared with the chosen
%   design?  From all the potential options, why did you select the documented design?
%   (LO\_Explores)
% \end{enumerate}


\end{document}