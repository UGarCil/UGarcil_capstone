\documentclass[12pt]{article}

\usepackage{enumitem,amssymb}
\newlist{todolist}{itemize}{2}
\setlist[todolist]{label=$\square$}
\usepackage{pifont}
\newcommand{\cmark}{\ding{51}}%
\newcommand{\xmark}{\ding{55}}%
\newcommand{\done}{\rlap{$\square$}{\raisebox{2pt}{\large\hspace{1pt}\cmark}}%
\hspace{-2.5pt}}
\newcommand{\wontfix}{\rlap{$\square$}{\large\hspace{1pt}\xmark}}

\begin{document}

\title{System Verification and Validation Plan Checklist}
\author{Spencer Smith}
\date{\today}

\maketitle

% Show an item is done by   \item[\done] Frame the problem
% Show an item will not be fixed by   \item[\wontfix] profit

\begin{itemize}

\item Follows writing checklist (full checklist provided in a separate document)
  \begin{todolist}
  \item \LaTeX{} points
  \item Structure
  \item Spelling, grammar, attention to detail
  \item Avoid low information content phrases
  \item Writing style
  \end{todolist}

\item Follows the template, all parts present
  \begin{todolist}
  \item Table of contents
  \item Pages are numbered
  \item Revision history included for major revisions
  \item Sections from template are all present
  \item Values of auxiliary constants are given (constants are used to improve
    maintainability and to increase understandability)
  \end{todolist}

\item Grammar, spelling, presentation
  \begin{todolist}
  \item No spelling mistakes (use a spell checker!)
  \item No grammar mistakes (review, ask someone else to review (at least a few
    sections))
  \item Paragraphs are structured well (clear topic sentence, cohesive)
  \item Paragraphs are concise (not wordy)
  \item No Low Information Content (LIC) phrases
    (\href{https://www.webpages.uidaho.edu/range357/extra-refs/empty-words.htm}{List
      of LIC phrases})
  \item All hyperlinks work
  \item Every figure has a caption
  \item Every table has a heading
  \item Symbolic names are used for quantities, rather than literal values
  \end{todolist}

\item LaTeX
  \begin{todolist}
  \item Template comments do not show in the pdf version, either by
    removing them, or by turning them off.
  \item References and labels are used so that maintenance is feasible
\end{todolist}

\item Overall qualities of documentation
  \begin{todolist}
\item Test cases include SPECIFIC input
\item Test cases include EXPLICIT output
\item Description over specification, when appropriate
\item Plans for what to do with description data (performance, usability, etc).
  This may involve saying what plots will be generated.
\item Plans to quantify error for scalar values using relative error
\item Plans to quantify error for vector and matrix values using a norm of an error
  vector (matrix)
\item Plans are feasible (can be accomplished with resources available)
\item Plans are ambitious enough for an A+ effort
\item Survey questions for usability survey are in an Appendix (if appropriate)
\item Plans for task based inspection, if appropriate
\item Very careful use of random testing
\item Specific programming language is listed
\item Specific linter tool is listed (if appropriate)
\item Specific coding standard is given
\item Specific unit testing framework is given
\item Investigation of code coverage measuring tools
\item Specific plans for Continuous Integration (CI), or an explanation that CI
  is not being done
\item Specific performance measuring tools listed (like Valgrind), if
  appropriate
\item Traceability between test cases and requirements is summarized (likely in
  a table)
\end{todolist}

\end{itemize}

\end{document}
