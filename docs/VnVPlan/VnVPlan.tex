\documentclass[12pt, titlepage]{article}
\usepackage{longtable}
\usepackage{booktabs}
\usepackage{tabularx}
\usepackage{hyperref}
\usepackage{array}     
\usepackage{multirow}  
\usepackage{amsmath}
\hypersetup{
    colorlinks,
    citecolor=blue,
    filecolor=black,
    linkcolor=red,
    urlcolor=green
}
\usepackage[round]{natbib}

\input{../Comments}
%% Common Parts

\newcommand{\progname}{SubLiMat} % PUT YOUR PROGRAM NAME HERE
\newcommand{\authname}{Uriel Garcilazo Cruz} % AUTHOR NAMES                  

\usepackage{hyperref}
    \hypersetup{colorlinks=true, linkcolor=blue, citecolor=blue, filecolor=blue,
                urlcolor=blue, unicode=false}
    \urlstyle{same}
                                


\begin{document}

\title{System Verification and Validation Plan for \progname{}} 
\author{\authname}
\date{\today}
	
\maketitle

\pagenumbering{roman}

\section*{Revision History}

\begin{tabularx}{\textwidth}{p{3cm}p{2cm}X}
\toprule {\bf Date} & {\bf Version} & {\bf Notes}\\
\midrule
Feb. 15th & 1.0 & First version\\
% Date 2 & 1.1 & Notes\\
\bottomrule
\end{tabularx}

% ~\\
% \wss{The intention of the VnV plan is to increase confidence in the software.
% However, this does not mean listing every verification and validation technique
% that has ever been devised.  The VnV plan should also be a \textbf{feasible}
% plan. Execution of the plan should be possible with the time and team available.
% If the full plan cannot be completed during the time available, it can either be
% modified to ``fake it'', or a better solution is to add a section describing
% what work has been completed and what work is still planned for the future.}

% \wss{The VnV plan is typically started after the requirements stage, but before
% the design stage.  This means that the sections related to unit testing cannot
% initially be completed.  The sections will be filled in after the design stage
% is complete.  the final version of the VnV plan should have all sections filled
% in.}

\newpage

\tableofcontents

% \listoftables
% \wss{Remove this section if it isn't needed}

% \listoffigures
% \wss{Remove this section if it isn't needed}

\newpage

\section{Symbols, Abbreviations, and Acronyms}

The following tables are extracted from the program's SGS documentation (Garcilazo-Cruz, 2025).
The tables are presented here for easy reference and cross-validation.


\renewcommand{\arraystretch}{1.2}
\begin{center}
  \begin{tabular}{l l l} 
    \toprule		
    \textbf{symbol} & \textbf{unit} & \textbf{HGVS}\\
    \midrule 
    *qa & alignment quality & fundamental unit of alignment quality
    \\
    bp & base pair & fundamental unit of genetic sequence length
    \\
    Kb & kilobase unit & one thousand base pairs
    \\
    \bottomrule
  \end{tabular}
\end{center}

\noindent 
\begin{longtable}{l l p{12cm}} \toprule
  \textbf{symbol} & \textbf{unit} & \textbf{description}\\
  \midrule 
    F & -- & Comparative alignment between two sequences
    \\
    g & qa & penalty associated with a gap in the alignment of two sequences
    \\
    $N_A$ & bp & Adenine nitrogenous base, element of the purine family of nucleotides with an amine group in Carbon 6 of its pyrimidine ring
    \\
    $N_C$ & bp & Cytosine nitrogenous base, element of the pyrimidine family of nucleotides with a no methylated carbons making part of its pyrimidine ring
    \\
    $N_G$ & bp & Guanine nitrogenous base, element of the purine family of nucleotides with an amine group on Carbon 2 and a carbonyl group on Carbon 6 of its pyrimidine ring
    \\
    $N_T$ & bp & Tymine nitrogenous base, element of the pyrimidine family of nucleotides with a methyl group in Carbon 5 of its pyrimidine ring
    \\
    $Q_{AB}$ & qa & A collection of base pairs representing a genetic sequence to be compared with another sequence $SEQ_A$
    \\
    S & qa & Substitution matrix used to score the alignment of two sequences
    \\
    $SEQ_A$ & Kb & A collection of base pairs representing a genetic sequence to be compared with another sequence $SEQ_B$
    \\
    $SEQ_B$ & Kb & A collection of base pairs representing a genetic sequence to be compared with another sequence $SEQ_A$
    \\
    SNP & bp & single nucleotide polymorphism; variation in a single base pair in DNA sequence
    \\
    $T_I$ & qa & A transition occurrying between nucleotides of the same nitrogenous base families
    \\
    $T_V$ & qa & A transition occurrying between nucleotides of different nitrogenous base families
    \\
  \bottomrule
\end{longtable}

% \begin{tabular}{l l} 
%   \toprule		
%   \textbf{symbol} & \textbf{description}\\
%   \midrule 
%   T & Test\\
%   \bottomrule
% \end{tabular}\\

% \wss{symbols, abbreviations, or acronyms --- you can simply reference the SRS
%   \citep{SRS} tables, if appropriate}

% \wss{Remove this section if it isn't needed}

\newpage

\pagenumbering{arabic}
This document serves as a systematic point of reference for the \progname{} software verification and validation process,
with the goal of thoroughly documenting the verification and validation goals of the scientific program.
The reader should refer to the SRS documentation for a detailed description of the software's theoretical framework, requirements
 and functionality. The document begins with the General information \autoref{sec:general-info}, followed by a verification plan and 
 a description of the system tests \autoref{sec:plan}. The document concludes with a unit test description \autoref{sec:unit-test-description}

\section{General Information}\label{sec:general-info}

\subsection{Summary}

This document reviews the verification and discusses the validation for the Substitution Matrix 
Benchmarking with Pariwise Sequence alignment (\progname{}) software. \progname{} is a program 
that describes the quality of genetic sequence alignments across different substitution matrices.

% \wss{Say what software is being tested.  Give its name and a brief overview of
%   its general functions.}

\subsection{Objectives}

The primary objective of the presented documentation is to provide a framework that 
enables a comparative and structured analysis on the confidence of the \progname{} software 
correctness, while also exposing potential boundaries in the software's usability, and 
demonstrating examples of adequate usability, helping build expectations and giving 
awareness of potential caveats.
Some objectives are out of the scope of the program's verification and validation analysis, at least 
in the documentation's first version, due to the limited amount of resources available for the project, 
including but not restricted to hardware and logistic constraints. The objectives outside of the scope 
of this documentation include a comprehensive testing of the software's validity given the lack of 
domain experts available. The verification and validation document will not include a comprehensive 
comparison with other software, due to the fundamental difference in the software's approach to
solving sequence alginments via heuristic algorithms, whereas \progname{} assumes the identification of the 
global optimum. Lastly, the verification of large datasets, including genomic-sized data, are out of the 
scope of the current testing protocol, as is is assumed the software works with coding genetic sequences 
commonly found in molecular biology.

% \wss{State what is intended to be accomplished.  The objective will be around
%   the qualities that are most important for your project.  You might have
%   something like: ``build confidence in the software correctness,''
%   ``demonstrate adequate usability.'' etc.  You won't list all of the qualities,
%   just those that are most important.}

% \wss{You should also list the objectives that are out of scope.  You don't have 
% the resources to do everything, so what will you be leaving out.  For instance, 
% if you are not going to verify the quality of usability, state this.  It is also 
% worthwhile to justify why the objectives are left out.}

% \wss{The objectives are important because they highlight that you are aware of 
% limitations in your resources for verification and validation.  You can't do everything, 
% so what are you going to prioritize?  As an example, if your system depends on an 
% external library, you can explicitly state that you will assume that external library 
% has already been verified by its implementation team.}

\subsection{Challenge Level and Extras}

The scope of verification and validation for \progname{} is considered to be 
basic, as the software is a scientific program that does not require a high level
of complexity in its verification and validation process, and there's a clearly 
defined set of validation tools that serve as a ground-truth to test the performance 
and accuracy of the software.

Additional (extra) features tackled by the project include the implementation of 
a verification software, which is assumed to mutate a sequence $SEQ_A$ from sequence
$SEQ_B$ to generate the real alignment and score, which can then be compared with
the alignment and score generated by the \progname{} software.

\subsection{Relevant Documentation}

The relevant documentation for \progname{} includes the Software Requirements Specification \citet{SRS},
the Module Guide (MG) and the Problem Statement. These elements can be found in the \href{https://github.com/UGarCil/UGarcil_capstone}{GitHub}  
repository for the project, and are available for consultation at any time.

The SRS document provides a detailed description of the software's requirements, including the
functional and non-functional requirements, and the theoretical framework that supports the software's
functionality. The MG document provides a detailed description of the software's architecture, including
the modules that compose the software, and the relationships between them. The Problem Statement provides
a brief overview of the software's purpose and functionality, and the motivation behind the software's
development.
% \wss{Reference relevant documentation.  This will definitely include your SRS
%   and your other project documents (design documents, like MG, MIS, etc).  You
%   can include these even before they are written, since by the time the project
%   is done, they will be written.  You can create BibTeX entries for your
%   documents and within those entries include a hyperlink to the documents.}



% \wss{Don't just list the other documents.  You should explain why they are relevant and 
% how they relate to your VnV efforts.}

\section{Plan}

The verification and validation plan for \progname{} is divided into several sections. 
First, there is a description of the verification and validation team, where the roles of
each team member are defined. Next, there is a description of the SRS verification plan, 
which includes the approaches that will be used to verify the software's requirements 
(both functional and non-functional). Then there is a description of the design verification
plan and the implementation verification plan, which describe the decisions and procedures 
that will be made to deploy testing.




% \wss{Introduce this section.  You can provide a roadmap of the sections to
%   come.}

\subsection{Verification and Validation Team}

\begin{table}[h]
  \begin{tabular}{|p{3cm}|p{2cm}|p{2.5cm}|p{6cm}|}
  \hline
  Name & Document & Role & Description \\
  \hline
  Dr. Spencer Smith & All & Instructor & Reviews documentation correctness and adherence to verification standards \\
  \hline
  Uriel Garcilazo Cruz & All & Author & Creator of the documentation, including SGS, MG1 and VnV \\
  \hline
  Junwei Lin & All & Domain Expert & Reviews documentation content and ensures the validity in the theory supported by each document \\
  \hline
  \end{tabular}
\end{table}

% \wss{Your teammates.  Maybe your supervisor.
%   You should do more than list names.  You should say what each person's role is
%   for the project's verification.  A table is a good way to summarize this information.}

\subsection{SRS Verification Plan}

The documentation for \progname{} will be reviewed by the team members, including the author,
using the following recipe:

\begin{enumerate}
  \item The author will review the documentation for correctness, adherence to verification and alignment to documentation standards.
  \item A new issue will be created for each version of the documentation on GitHub, and shared with the collaborators Dr. Spencer Smith \\
   and Junwei Lin.
  \item The domain expert will review the documentation for correctness and ensure the validity of the theory supported by each document.
  \item The instructor will review the documentation for correctness and alignment with documentation standards.
  \item Author will address the issues and make the necessary changes to the documentation.
\end{enumerate}

% \wss{List any approaches you intend to use for SRS verification.  This may
%   include ad hoc feedback from reviewers, like your classmates (like your
%   primary reviewer), or you may plan for something more rigorous/systematic.}

% \wss{If you have a supervisor for the project, you shouldn't just say they will
% read over the SRS.  You should explain your structured approach to the review.
% Will you have a meeting?  What will you present?  What questions will you ask?
% Will you give them instructions for a task-based inspection?  Will you use your
% issue tracker?}

% \wss{Maybe create an SRS checklist?}

\subsection{Design Verification Plan}

The design verification plan for \progname{} will be reviewed by the team members, including the author,
using a dynamic verification approach by the main author Uriel Garcilazo-Cruz, and assisted in the 
verification and test accuracy by the domain expert Junwei Lin. The instructor Dr. Spencer Smith will 
review the documentation to determine correctness and adherence to verification standards.

% \wss{Plans for design verification}

% \wss{The review will include reviews by your classmates}

% \wss{Create a checklists?}

\subsection{Verification and Validation Plan Verification Plan}

% \wss{The verification and validation plan is an artifact that should also be
% verified.  Techniques for this include review and mutation testing.}

The verification and validation plan for \progname{} will be reviewed systematically through the following process:

\begin{enumerate}
  \item The author (Uriel Garcilazo Cruz) will create and validate the initial VnV plan.
  \item A new issue will be created on GitHub for each version of the VnV plan.
  \item Domain expert (Junwei Lin) will review the documentation and provide feedback through GitHub issues
  \item Dr. Spencer Smith will conduct the final review of the VnV plan as the instructor.
  \item The author will address all issues and incorporate suggested changes.
  \item All reviews will be guided by the VnV Checklist (\href{https://github.com/UGarCil/UGarcil_capstone/blob/main/docs/Checklists/VnV-Checklist.pdf}{VnV-Checklist.pdf}).
\end{enumerate}
The progress of this review process will be tracked through GitHub issues, allowing for transparent communication and documentation of all feedback and revisions.

% \wss{The review will include reviews by your classmates}

% \wss{Create a checklists?}

\subsection{Implementation Verification Plan}

To verify \progname{}'s implementation, the team will follow the following steps:
\begin{enumerate}
  \item Static verification:
  \begin{itemize}
    \item Author (Uriel Garcilazo-Cruz) will validate and create the first version of the code.
    \item Using pylint to check for code quality and adherence to coding standards in the
    programming language Python.
  \end{itemize}
  
  \item Dynamic verification:
  \begin{itemize}
    \item The test cases indicated in \autoref{unittd:main} will be used to verify the implementation of the software.
    \item Using pytest to carry out the tests and verify the software's functionality.
  \end{itemize}

\end{enumerate}

% \wss{You should at least point to the tests listed in this document and the unit
%   testing plan.}

% \wss{In this section you would also give any details of any plans for static
%   verification of the implementation.  Potential techniques include code
%   walkthroughs, code inspection, static analyzers, etc.}

% \wss{The final class presentation in CAS 741 could be used as a code
% walkthrough.  There is also a possibility of using the final presentation (in
% CAS741) for a partial usability survey.}

\subsection{Automated Testing and Verification Tools}

The automated testing will take place using the pytest framework, which will be used to run the test cases
in combination with continuous integration allowed by GitHub Actions. The test cases will be written in Python,
and will be used to verify the software's functionality.

% \wss{What tools are you using for automated testing.  Likely a unit testing
%   framework and maybe a profiling tool, like ValGrind.  Other possible tools
%   include a static analyzer, make, continuous integration tools, test coverage
%   tools, etc.  Explain your plans for summarizing code coverage metrics.
%   Linters are another important class of tools.  For the programming language
%   you select, you should look at the available linters.  There may also be tools
%   that verify that coding standards have been respected, like flake9 for
%   Python.}

% \wss{If you have already done this in the development plan, you can point to
% that document.}

% \wss{The details of this section will likely evolve as you get closer to the
%   implementation.}

\subsection{Software Validation Plan}

A validation plan for \progname{} is beyond the scope of this document,
given the limited amount of domain experts available to perform such validation, 
and the time available to complete the project's verification.

% \wss{If there is any external data that can be used for validation, you should
%   point to it here.  If there are no plans for validation, you should state that
%   here.}

% \wss{You might want to use review sessions with the stakeholder to check that
% the requirements document captures the right requirements.  Maybe task based
% inspection?}

% \wss{For those capstone teams with an external supervisor, the Rev 0 demo should 
% be used as an opportunity to validate the requirements.  You should plan on 
% demonstrating your project to your supervisor shortly after the scheduled Rev 0 demo.  
% The feedback from your supervisor will be very useful for improving your project.}

% \wss{For teams without an external supervisor, user testing can serve the same purpose 
% as a Rev 0 demo for the supervisor.}

% \wss{This section might reference back to the SRS verification section.}

\section{System Tests}

% \wss{There should be text between all headings, even if it is just a roadmap of
% the contents of the subsections.}

\subsection{Tests for Functional Requirements}\label{func-req}

In agreement with the functional requirements stated in the SRS document \citet{SRS},
the following tests will be performed to verify the software's functionality. R1 amd R2 
functional requirements are related to the \progname{} inputs, and R3, R4 and R5 are 
related to the software's outputs.


% \wss{Subsets of the tests may be in related, so this section is divided into
%   different areas.  If there are no identifiable subsets for the tests, this
%   level of document structure can be removed.}

% \wss{Include a blurb here to explain why the subsections below
%   cover the requirements.  References to the SRS would be good here.}

\subsubsection{Input Tests: Valid Inputs}

The SRS document \citet{SRS} refers to the functional requirement R1 as the main 
descriptor on the software's inputs, and R2 for the construction of a comparative matrix 
used in \citet{NEEDLEMAN1970443}.
% \wss{It would be nice to have a blurb here to explain why the subsections below
%   cover the requirements.  References to the SRS would be good here.  If a section
%   covers tests for input constraints, you should reference the data constraints
%   table in the SRS.}
		
% \paragraph{Input Test: Valid Inputs}



\begin{enumerate}

\begin{table}[h]
  \begin{tabular}{|c|c c|c c|}
  \hline 
    & \multicolumn{2}{c|}{Input (bp)} & \multicolumn{2}{c|}{Output} \\
  \cline{2-5}
  ID & $SEQ_A$ & $SEQ_B$ & valid? & Error Message \\
  \hline
  TC-SubLiMat-1-1 & ATCG & ATCG & Y & NONE \\
  \hline
  TC-SubLiMat-1-2 & " " & ATCG & N & Sequence length bigger than zero \\
  \hline
  TC-SubLiMat-1-3 & ATCG & " " & N & Sequence length bigger than zero \\
  \hline
  TC-SubLiMat-1-4 & " " & " " & N & Sequence length bigger than zero \\
  \hline
  TC-SubLiMat-1-5 & ATCG & A\_CG & Y & NONE \\
  \hline
  TC-SubLiMat-1-6 & A & ATCGGG & Y & NONE \\
  \hline
  TC-SubLiMat-1-7 & A & T & Y & NONE \\
  \hline
  TC-SubLiMat-1-8 & RTGA & ATGA & N & Sequence contains non DNA symbol \\
  \hline
  \end{tabular}
  \caption{Test case inputs and outputs}
  \label{tab:test-case-1}
\end{table}

\item{Functional Input Tests - Properties of genetic sequence data (R1)\\}

Control: Automatic
					
Initial State: Hyperparameter defined during runtime
					
Input: Inputs from Table~\ref{tab:test-case-1}
					
Output: Return an error message with a detailed description of the sequence(s)
creating the issue, or a list of numbers representing the alignment score(s) for the
pairwise sequence alignment(s) of the input sequences $SEQ_A$ and $SEQ_B$.

% \wss{The expected result for the given inputs.  Output is not how you
% are going to return the results of the test.  The output is the expected
% result.}

Test Case Derivation: Evaluates the software's ability to handle different types of input data 
that represent genetic sequences, including valid and invalid sequences, and sequences with 
different lengths. The system returns error messages when the physical constrains of the data are 
violated, in cases where the lenght of the sequence is zero, or when the sequence contains non-DNA
symbols.

% \wss{Justify the expected value given in the Output field}

How test will be performed: Automatically using the pytest framework.
					


\item{Functional Input Tests - Input Tests: Valid matrices of F(R2)\\}
\begin{table}[h]
  \begin{tabular}{|c|c c|c c|}
  \hline 
    & \multicolumn{2}{c|}{Input (bp)} & \multicolumn{2}{c|}{Output} \\
  \cline{2-5}
  ID & $SEQ_{A_k=1}^n$ & $SEQ_{B_k=1}^m$ & Valid? & Error Message \\
  \hline
  TC-SubLiMat-2-1 & $n=10$ & $m=10$ & Y & NONE \\
  \hline
  TC-SubLiMat-2-2 & $n=100$ & $m=10$ & Y & NONE \\
  \hline
  TC-SubLiMat-2-3 & $n=10$ & $m=100$ & Y & NONE \\
  \hline
  TC-SubLiMat-2-4 & $n=500$ & $m=500$ & Y & NONE \\
  \hline
  TC-SubLiMat-2-5 & $n=5,000$ & $m=5,000$ & Y & NONE \\
  \hline
  TC-SubLiMat-2-6 & $n \geq 5,001$ & $m \geq 5,001$ & N & Sequence length exceeds maximum \\
  \hline
  \end{tabular}
  \caption{Test cases for sequence length validation}
  \label{tab:test-case-2}
 \end{table}

Control: Automatic
					
Initial State: Parameter defined during runtime
					
Input: Valid inputs from Table~\ref{tab:test-case-1} with cases taken from elements from 
Table~\ref{tab:test-case-2}

					
Output: Return a matrix with the alignment scores for the pairwise sequence alignment of the input sequences
 $SEQ_A$ and $SEQ_B$. 
% \wss{The expected result for the given inputs}

Test Case Derivation: Evaluates the software's ability to construct a comparison matrix for the alignment of 
two genetic sequences and returns a null value of N/A for matrices of size greater than 5,000 elements.
% \wss{Justify the expected value given in the Output field}

How test will be performed: Automatically using the pytest framework.

% ####################################################################
\newpage

\item{Functional Input Tests - Functional Input Tests - Input Tests: Valid matrices of S (R3)\\\\}

\begin{table}[h]
  \begin{tabular}{|c|c|c c|}
  \hline 
    & Input Matrix & \multicolumn{2}{c|}{Expected Output} \\
  \cline{2-4}
  ID & Dimensions \& Content & Valid? & Error Message \\
  \hline
  TC-SubMat-3-1 & $n \times n, n=4$ & Y & NONE \\
  \hline
  TC-SubMat-3-2 & $n \times m, n=4, m=5$ & N & Matrix must be square \\
  \hline
  TC-SubMat-3-3 & $n \times n, n=4$ , non-numeric & N & Matrix entries must be numeric \\
  \hline
  TC-SubMat-3-4 & $n \times n, n=3$ & N & Matrix must be 4x4 for nucleotides \\
  \hline
  TC-SubMat-3-5 & $n \times n, n=4$ asymmetric & N & Matrix must be symmetric \\
  \hline
  \end{tabular}
  \caption{Test cases for substitution matrix validation. A valid substitution matrix must be square with dimensions 4x4 for nucleotides,
  contain only numeric values, and be symmetric for reciprocal comparisons between elements.}
  \label{tab:test-case-3}
\end{table}

Control: Automatic
					
Initial State: Hyperparameter defined by the program 
					
Input: Inputs from Table~\ref{tab:test-case-3} 

					
Output: Return a boolean value indicating the validity of the input matrix $S$.
% \wss{The expected result for the given inputs}

Test Case Derivation: Evaluates the software's ability to validate the input substitution matrix $S$ for the
alignment of two genetic sequences, ensuring the matrix is square, contains only numeric values, and is symmetric.

% \wss{Justify the expected value given in the Output field}

How test will be performed: Automatically using the pytest framework.

% ################################################################
\end{enumerate}
\subsubsection{Output Tests: Coherence in the alignment}
\begin{enumerate}

\item{Functional Output Tests - Coherence in the alignment (R4 \& R5)\\}

Control: Automatic
					
Initial State: Pending output conditional to valid inputs
					
Input: Set of calculated alignment scores given a matrix F
					
Output: An error message that appears when the maximum length of a sequence as input is reached, and nothing otherwise.
 $SEQ_A$ and $SEQ_B$.
% \wss{The expected result for the given inputs}

Test Case Derivation: Evaluates the software's ability to process alignment scores, and build a comparison matrix F in 
a reasonable amount of time.

% \wss{Justify the expected value given in the Output field}

How test will be performed: Automatically using the pytest framework.


\end{enumerate}


\subsection{Tests for Nonfunctional Requirements}

As described in \progname{} SRS document \citet{SRS}, there are five non-functional requirements
that will be tested in the software. The non-functional requirements for accuracy will just reference the
appropriate functional tests from above.


% \wss{The nonfunctional requirements for accuracy will likely just reference the
%   appropriate functional tests from above.  The test cases should mention
%   reporting the relative error for these tests.  Not all projects will
%   necessarily have nonfunctional requirements related to accuracy.}

% \wss{For some nonfunctional tests, you won't be setting a target threshold for
% passing the test, but rather describing the experiment you will do to measure
% the quality for different inputs.  For instance, you could measure speed versus
% the problem size.  The output of the test isn't pass/fail, but rather a summary
% table or graph.}

% \wss{Tests related to usability could include conducting a usability test and
%   survey.  The survey will be in the Appendix.}

% \wss{Static tests, review, inspections, and walkthroughs, will not follow the
% format for the tests given below.}

% \wss{If you introduce static tests in your plan, you need to provide details.
% How will they be done?  In cases like code (or document) walkthroughs, who will
% be involved? Be specific.}

\subsubsection{Nonfunctional: Performance (NFR5 in \citet{SRS})}
		
\paragraph{Performance}

\begin{enumerate}

\item{Nonfunctional Tests: Performance}


\begin{table}[h]
  \begin{tabular}{|c|c c|c c|}
  \hline 
    & \multicolumn{2}{c|}{Input (bp)} & \multicolumn{2}{c|}{Expected Performance} \\
  \cline{2-5}
  ID & $SEQ_{A_k=1}^n$ & $SEQ_{B_k=1}^m$ & Matrix Size & Est. Time (s) \\
  \hline
  TC-SubLiMat-4-1 & $n=10$ & $m=10$ & $100$ & $< 0.1$ \\
  \hline
  TC-SubLiMat-4-2 & $n=100$ & $m=10$ & $1,000$ & $< 1.0$ \\
  \hline
  TC-SubLiMat-4-3 & $n=10$ & $m=100$ & $1,000$ & $< 1.0$ \\
  \hline
  TC-SubLiMat-4-4 & $n=500$ & $m=500$ & $250,000$ & $\sim 25$ \\
  \hline
  TC-SubLiMat-4-5 & $n=5,000$ & $m=5,000$ & $25,000,000$ & $> 300$ \\
  \hline
  TC-SubLiMat-4-3 & $n \geq 5,001$ & $m \geq 5,001$ & N/A & $< 0.1$ \\
  \hline
  \end{tabular}
  \caption{Test cases for estimated time of completion given the size of the comparison matrix}
  \label{tab:test-case-2}
\end{table}


Initial State: Pending output conditional to valid inputs
					
Input: Set of calculated alignment scores given a matrix F
					
Output: A float value representing the amount of time taken to calculate the alignment scores for the input sequences
 $SEQ_A$ and $SEQ_B$.
% \wss{The expected result for the given inputs}

Test Case Derivation: Evaluates the software's ability to execute, in a commercial computer, the alignment of two genetic sequences
 $SEQ_A$ and $SEQ_B$.

% \wss{Justify the expected value given in the Output field}

How test will be performed: Automatically using the pytest framework.

\end{enumerate}

\subsubsection{Nonfunctional: Portability (NFR4 in \citet{SRS})}
\paragraph{Portability}

\begin{enumerate}
  \item{Nonfunctional Tests: Portability}
  \begin{table}[h]
    \centering
    \begin{tabular}{|c|c|c|}
      \hline
      ID & Operating System & Expected Result \\
      \hline
      TC-SCEC-5-1 & Windows & Pass \\
      \hline
      TC-SCEC-5-2 & Linux & Pass \\
      \hline
      TC-SCEC-5-3 & macOS & Pass \\
      \hline
    \end{tabular}
    \caption{Test cases for system compatibility across different platforms}
    \label{tab:test-case-portability}
  \end{table}

Initial State: Fresh system installation with no prior software components

Input: Software installation package

Output: Successful running system over all platforms to verify portability of the software, and 
summarized in a table.

Test Case Derivation: Evaluates the software's ability to install and run consistently across different operating systems while maintaining functionality and user experience.

How test will be performed: Code developer (Uriel Garcilazo Cruz) will try to install and run whole software in different operating systems.
\end{enumerate}
  

% ####################################################################

\subsubsection{Nonfunctional: Usability (NFR2 in \citet{SRS})}
\paragraph{Usability}

\begin{enumerate}

\begin{table}[h]
  \centering
  \begin{tabular}{|c|p{11cm}|c|}
      \hline
      No. & Question & Answer \\
      \hline
      1. & How many cores does your computer have? & \\
      \hline
      2. & Are you familiar with the use of Python? & \\
      \hline
      3. & Are the manual specifications easy to follow? & \\
      \hline
      4. & Did you have Python installed in your machine before? & \\
      \hline
      5. & Did you encounter any issues running the file from terminal? & \\
      \hline
      6. & Did you encounter any issues loaading your genetic sequences? & \\
      \hline
      7. & Is the output table easy to understand? & \\
      \hline
  \end{tabular}
  \caption{TC-SCEC-5 - Usability test survey}
  \label{tab:usability-survey}
\end{table}

\item{Nonfunctional Tests: Usability}

Type: Manual

Input: None

Initial State: None 

Output: Survey with user responses

Test Case Derivation: Evaluates how easy it is for the user to interact with the software, and how intuitive the software is to use.

How test will be performed: An online survey will be created and shared with potential users of the software, 
including domain experts and, if possible, students in the field of molecular biology or collaborators that meet the 
user specifications found in the SRS document \citet{SRS}.
\end{enumerate}
  

% ####################################################################

\newpage
\subsection{Traceability Between Test Cases and Requirements}

\begin{table}[h]
  \centering
  \begin{tabular}{|c|c|c|c|c|c|c|c|c|}
  \hline
   & R1 & R2 & R3 & R4/R5 & NFR2 & NFR4 & NFR5 \\
  \hline
  TC-SubLiMat-1 & X &   &   &   &   &   &   \\
  \hline
  TC-SubLiMat-2 &   & X &   &   &   &   &   \\
  \hline
  TC-SubMat-3   &   &   & X &   &   &   &   \\
  \hline
  TC-SubLiMat-4 &   &   &   & X &   &   & X \\
  \hline
  TC-SCEC-5     &   &   &   &   & X & X &   \\
  \hline
  \end{tabular}
  \caption{Traceability between test cases and requirements}
  \label{tab:traceability}
\end{table}

\section{Unit Test Description}\label{unittd:main}

The program \progname{} is composed of the following modules:

\begin{itemize}
  \item \textbf{constants.py}: Ensures proper integration with arguments received from user and initializes constants and hyperparameters in the program
  \item \textbf{pairwise\_alignment.py}: Responsible of constructing the comparison matrix $F$ and implementing the Needleman-Wunsch algorithm
  \item \textbf{main.py}: Integrates calculation of alignment scores over all substitution matrices and handles formatting
  \end{itemize}
  

% \wss{This section should not be filled in until after the MIS (detailed design
%   document) has been completed.}

% \wss{Reference your MIS (detailed design document) and explain your overall
% philosophy for test case selection.}  

% \wss{To save space and time, it may be an option to provide less detail in this section.  
% For the unit tests you can potentially layout your testing strategy here.  That is, you 
% can explain how tests will be selected for each module.  For instance, your test building 
% approach could be test cases for each access program, including one test for normal behaviour 
% and as many tests as needed for edge cases.  Rather than create the details of the input 
% and output here, you could point to the unit testing code.  For this to work, you code 
% needs to be well-documented, with meaningful names for all of the tests.}

\subsection{Unit Testing Scope}

The unit testing for \progname{} will focus on the following modules:
\begin{itemize}
  \item \textbf{pairwise\_alignment.py}
\end{itemize}

The core of the algorithm is implemented in the \textbf{pairwise\_alignment.py} module, 
which is responsible for constructing the comparison matrix $F$ and implementing the Needleman-Wunsch algorithm.
 This module is the most critical part of the software, and the unit tests will focus on verifying the correctness 
 of the alignment scores calculated by the algorithm. 

The modules \textbf{constants.py} and \textbf{main.py} will not be tested in this document, 
as they are not critical to the software's functionality, serving in the initialization and  
calling of the algorithm, respectively.

% \wss{What modules are outside of the scope.  If there are modules that are
%   developed by someone else, then you would say here if you aren't planning on
%   verifying them.  There may also be modules that are part of your software, but
%   have a lower priority for verification than others.  If this is the case,
%   explain your rationale for the ranking of module importance.}

\subsection{Tests for Functional Requirements}

All of the tests for the functional requirements are described in the section \autoref{func-req}, and 

% \wss{Most of the verification will be through automated unit testing.  If
%   appropriate specific modules can be verified by a non-testing based
%   technique.  That can also be documented in this section.}

\subsubsection{Module 1}

% \wss{Include a blurb here to explain why the subsections below cover the module.
%   References to the MIS would be good.  You will want tests from a black box
%   perspective and from a white box perspective.  Explain to the reader how the
%   tests were selected.}

% \begin{enumerate}

% \item{test-id1\\}

% Type: \wss{Functional, Dynamic, Manual, Automatic, Static etc. Most will
%   be automatic}
					
% Initial State: 
					
% Input: 
					
% Output: \wss{The expected result for the given inputs}

% Test Case Derivation: \wss{Justify the expected value given in the Output field}

% How test will be performed: 
					
% \item{test-id2\\}

% Type: \wss{Functional, Dynamic, Manual, Automatic, Static etc. Most will
%   be automatic}
					
% Initial State: 
					
% Input: 
					
% Output: \wss{The expected result for the given inputs}

% Test Case Derivation: \wss{Justify the expected value given in the Output field}

% How test will be performed: 

% \item{...\\}
    
% \end{enumerate}

% \subsubsection{Module 2}

% ...

\subsection{Tests for Nonfunctional Requirements}

Tests for nonfunctional requirements are beyond the scope of the unit testing for \progname{}.


% \wss{If there is a module that needs to be independently assessed for
%   performance, those test cases can go here.  In some projects, planning for
%   nonfunctional tests of units will not be that relevant.}

% \wss{These tests may involve collecting performance data from previously
%   mentioned functional tests.}

% \subsubsection{Module ?}
		
% \begin{enumerate}

% \item{test-id1\\}

% Type: \wss{Functional, Dynamic, Manual, Automatic, Static etc. Most will
%   be automatic}
					
% Initial State: 
					
% Input/Condition: 
					
% Output/Result: 
					
% How test will be performed: 
					
% \item{test-id2\\}

% Type: Functional, Dynamic, Manual, Static etc.
					
% Initial State: 
					
% Input: 
					
% Output: 
					
% How test will be performed: 

% \end{enumerate}

% \subsubsection{Module ?}

% ...

% \subsection{Traceability Between Test Cases and Modules}

% \wss{Provide evidence that all of the modules have been considered.}
				
\bibliographystyle{plainnat}

\bibliography{../../refs/References}

\newpage

% \section{Appendix}

% This is where you can place additional information.

% \subsection{Symbolic Parameters}

% The definition of the test cases will call for SYMBOLIC\_CONSTANTS.
% Their values are defined in this section for easy maintenance.

% \subsection{Usability Survey Questions?}

% \wss{This is a section that would be appropriate for some projects.}

% \newpage{}
% \section*{Appendix --- Reflection}

% \wss{This section is not required for CAS 741}

% The information in this section will be used to evaluate the team members on the
% graduate attribute of Lifelong Learning.

% \input{../Reflection.tex}

% \begin{enumerate}
%   \item What went well while writing this deliverable? 
%   \item What pain points did you experience during this deliverable, and how
%     did you resolve them?
%   \item What knowledge and skills will the team collectively need to acquire to
%   successfully complete the verification and validation of your project?
%   Examples of possible knowledge and skills include dynamic testing knowledge,
%   static testing knowledge, specific tool usage, Valgrind etc.  You should look to
%   identify at least one item for each team member.
%   \item For each of the knowledge areas and skills identified in the previous
%   question, what are at least two approaches to acquiring the knowledge or
%   mastering the skill?  Of the identified approaches, which will each team
%   member pursue, and why did they make this choice?
% \end{enumerate}

\end{document}