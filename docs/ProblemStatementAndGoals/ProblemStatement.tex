\documentclass{article}

\usepackage{tabularx}
\usepackage{booktabs}

\title{Problem Statement and Goals\\Substitution-Matrix benchmarking with pairwise sequence alignment}

\author{\\Uriel Garcilazo Cruz}

\date{January 18th, 2025}

%% Comments

\usepackage{color}

\newif\ifcomments\commentstrue %displays comments
%\newif\ifcomments\commentsfalse %so that comments do not display

\ifcomments
\newcommand{\authornote}[3]{\textcolor{#1}{[#3 ---#2]}}
\newcommand{\todo}[1]{\textcolor{red}{[TODO: #1]}}
\else
\newcommand{\authornote}[3]{}
\newcommand{\todo}[1]{}
\fi

\newcommand{\wss}[1]{\authornote{blue}{SS}{#1}} 
\newcommand{\plt}[1]{\authornote{magenta}{TPLT}{#1}} %For explanation of the template
\newcommand{\an}[1]{\authornote{cyan}{Author}{#1}}

%% Common Parts

\newcommand{\progname}{ProgName} % PUT YOUR PROGRAM NAME HERE
\newcommand{\authname}{Team \#, Team Name
\\ Student 1 name
\\ Student 2 name
\\ Student 3 name
\\ Student 4 name} % AUTHOR NAMES                  

\usepackage{hyperref}
    \hypersetup{colorlinks=true, linkcolor=blue, citecolor=blue, filecolor=blue,
                urlcolor=blue, unicode=false}
    \urlstyle{same}
                                


\begin{document}

\maketitle

\begin{table}[hp]
\caption{Revision History} \label{TblRevisionHistory}
\begin{tabularx}{\textwidth}{llX}
\toprule
\textbf{Date} & \textbf{Developer(s)} & \textbf{Change}\\
\midrule
18 January & Uriel Garcilazo Cruz & Document's first release\\
18 January & Uriel Garcilazo Cruz & Title's correction\\
\bottomrule
\end{tabularx}
\end{table}

\section{Problem Statement}

Substitution matrices make one of the axioms on which to elaborate hypotheses in comparative biology.
There are many substitution matrices used in the literature, 
and their effects in different types of sequences is not always easy to evaluate through benchmarking.

\subsection{Problem}

Aligning DNA strands from two individuals or species helps revealing past evolutionary events between them.
However, finding empirical values for the rate at which one nucleotide or aminoacid changes into another is difficult,
because any evidence of substitutions that may have occurred as intermediate stages gets erased by new mutations.
A square matrix that encodes the substitution rates among nucleotides or aminoacids is called a substitution matrix.
Although multiple substitution matrices have been proposed, and subsequently adopted as a standard for alignment \cite{Altschul1991,Mount2008},
it is of the utmost importance to determine how a given substitution matrix may impact the quality of an alignment.
For this it is useful to use methods of pairwise alignment \cite{NEEDLEMAN1970443} that ensure the best global score between two sequences.
This program enables the benchmarking of substitution matrices to determine their effects in a pairwise alignment.


\subsection{Inputs and Outputs}

Given two sequences of DNA, determine the effects that commonly used substitution matrices have in the quality of their alignment.

\subsection{Stakeholders}

Evolutionary Biologists and researchers with genomic or protein data,
interested in the effects of hyperparameters in the quality of their research.
\subsection{Environment}

Linux terminal recommended\\
Windows 10 or higher is recommended.\\
MacOS Sierra or later is recommended

\section{Goals}

\begin{itemize}
    \item Gives a way to measure the effects that a substitution matrix has over the quality of an alignment.
    \item Yields a comparative tool to evaluate the effects of diverse substitution matrices.
    \item Provides a benchmarking tool to evaluate the utility of matrices found in the literature.
\end{itemize}

\section{Stretch Goals}

\begin{itemize}
    \item By finding the best possible alignment, the data can be used to train machine learning algorithms.
    \item framework could be extended to optimize substitution matrices for specific sequence types.
    \item The analysis could help identify which matrix elements are most important for accurate alignment.
\end{itemize}

\newpage

\bibliographystyle {plain}
\bibliography {srs}

% A Python implementation of the core for this algorithm, called Needleman-Wunsch for pairwaise alignment can be found at:
% https://urielgarcilazo.com/tutorials/1_DynamicProgramming.html 

\end{document}